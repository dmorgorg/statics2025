\newcommand{\DL}[9][1]{
  \def\forcedown{#1} % defaults to 1, force is downward
  \def\tl{#2} % top left, a coordinate
  \def\tr{#3} % top right. a coordinate
  \def\b{#4} % anywhere along the baseline (before any rotation), a coordinate 
  \def\lfill{#5} % background fill color
  \def\stroke{#6} % drawing color
  \def\spaces{#7} % number of spaces between arrows 
  \def\llinewidth{#8}
  \def\tiplength{#9}

  \gettikzxy{(\tl)}{\tlx}{\tly}
	\gettikzxy{(\tr)}{\trx}{\try}
	\gettikzxy{(\b)}{\bx}{\by}
  \pgfmathparse{abs(\try-\by)} \let\rlength\pgfmathresult
  \pgfmathparse{abs(\tly-\by)} \let\llength\pgfmathresult

  \fill[\lfill] (\tlx, \tly)--(\trx, \try)--(\trx, \by)--(\tlx, \by);
  \draw[\stroke, line cap = round, line width = \llinewidth mm] (\tl)--(\tr);
  
  % no empty lines in \tikzmath!
  \tikzmath{
    % Calculate the width of the load, and the spacing between arrows
    % Also, calculate the difference in length between adjacent arrows.
    \dx = \trx - \tlx; % width of dist load
    \dx = \dx / \spaces; % space between arrows
    \dy = \try - \tly; % difference between two load values
    \dy = \dy / \spaces; % difference between arrow-line lengths
    %    
    if \forcedown == 1 then {       
			for \i in {0,...,\spaces} {	
        \starty = \tly+\i*\dy;
        \length = \starty-\by;
        % in \tikzmath, drawing commands are enclosed in { }; 
        {
          \begin{scope}          
            \clip (\tlx,\tly) -- (\trx, \try) -- (\trx,\by) --(\tlx,\by);
            \draw[\stroke, line width = \llinewidth mm, -{Latex[length=\tiplength]}](\tlx+\i*\dx, \starty pt)-- +(270: \length pt);
          \end{scope}
        };
			};
    } else {
      for \i in {0,...,\spaces} {	
        \starty = \tly+\i*\dy;
        \length = \starty-\by;			
				{
          \begin{scope}          
            \clip (\tlx,\tly) -- (\trx, \try) -- (\trx,\by) --(\tlx,\by);
            \draw[\stroke, line width = \llinewidth mm, {Latex[length=\tiplength]}-](\tlx+\i*\dx, \starty pt)-- +(270: \length pt);
          \end{scope}          
        };
			};      
    };
    if \forcedown == 1 then {
      if \rlength > \tiplength then {
        {\draw[\stroke, line width = \llinewidth mm, -{Latex[length=\tiplength]}] (\trx, \try)--(\trx, \by);};
      } else {
         {\draw[\stroke, line width = \llinewidth mm] (\trx, \try)--(\trx, \by);};
      };    
      if \llength > \tiplength then {
        {\draw[\stroke, line width = \llinewidth mm, -{Latex[length=\tiplength]}] (\tlx, \tly)--(\tlx, \by);};
      } else {
        {\draw[\stroke, line width = \llinewidth mm] (\tlx, \tly)--(\tlx, \by);};
      };
    } else {
      if \rlength > \tiplength then {
        {\draw[\stroke, line width = \llinewidth mm, {Latex[length=\tiplength]}-] (\trx, \try)--(\trx, \by);};
      } else {
        {\draw[\stroke, line width = \llinewidth mm] (\trx, \try)--(\trx, \by);};
      };    
      if \llength > \tiplength then {
        {\draw[\stroke, line width = \llinewidth mm, {Latex[length=\tiplength]}-] (\tlx, \tly)--(\tlx, \by);};
      } else {
        {\draw[\stroke, line width = \llinewidth mm] (\tlx, \tly)--(\tlx, \by);};
      };
    };    
  } % \end tikzmath environment
} % end of \DL definition