% !TEX root = ../Beamer/statikz/statikz.tex

% \Ring{A}{outer color}{inner color}{outer radius}{inner radius}{line width}
\newcommand{\Ring}[6]{
	\def\lpin{#1}
	\def\lfill{#2}
	\def\ldraw{#3}
	\def\outerr{#4}
	\def\innerr{#5}
	\def\lwidth{#6}

	\begin{scope}

		\makeatletter
		\providecommand{\gettikzxy}[3]{%
			\tikz@scan@one@point\pgfutil@firstofone#1\relax
			\edef#2{\the\pgf@x}%
			\edef#3{\the\pgf@y}%
		}
		\makeatother

		\gettikzxy{(\lpin)}{\cx}{\cy}
		\pgfdeclareradialshading{ring}{\pgfpoint{0cm}{0cm}}
		{
			color(0cm)=(black);
			color(0.5cm)=(\lfill);
			color(.65cm)=(\ldraw);
			color(1cm)=(\lfill)
		}
		% \pgfuseshading{ring}



	\end{scope}


\begin{scope}[even odd rule]
	% \draw (\lpin) circle (\innerr);
	\filldraw[shading=ring, fill=\lfill, draw=\ldraw, line width=\lwidth] (\lpin) circle (\outerr cm)
		(\lpin) circle (\innerr);
		\draw[black, line width = \lwidth mm] (\lpin) circle (\innerr cm);
		\draw[black, line width = \lwidth mm] (\lpin) circle (\outerr cm);
\end{scope}


}
