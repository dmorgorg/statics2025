%\Member{startpt}{endpt}{outer fill color}{inner fill color}{stroke}{height}{radius}{linewidth}
\providecommand{\Meme}[8]{
  \coordinate(start) at (#1);
  \coordinate(end) at (#2);
  \edef\ofill{#3}%
  \edef\ifill{#4}%
  \edef\stroke{#5}%
  \edef\height{#6} % cm
  \edef\radius{#7} % cm, should be half \height or less
  \edef\linewidth{#8} % mm
\
  


  \coordinate(delta) at ($ (end)-(start) $);
  \gettikzxy{(delta)}{\dx}{\dy}
  \gettikzxy{(start)}{\sx}{\sy}
  \gettikzxy{(end)}{\ex}{\ey}
  \pgfmathparse{veclen(\dx, \dy)} \let\length\pgfmathresult
  \pgfmathparse{\height*28.435} \let\heightpt\pgfmathresult
  \pgfmathparse{\heightpt/\length} \let\ratio\pgfmathresult
  \pgfmathparse{1/\ratio} \let\inverse\pgfmathresult
  

  \pgfmathparse{\dx==0}%
  % \ifnum low-level TeX for integers
  \ifnum\pgfmathresult=1 % \dx == 0
    \pgfmathsetmacro{\rot}{\dy > 0 ? 90 : -90}
  \else
    \pgfmathsetmacro{\rot}{\dx > 0 ? atan(\dy / \dx) : 180 + atan(\dy / \dx)}
  \fi

  \pgfmathparse{round(mod(abs(\rot),90))} \let\tmp\pgfmathresult
  \pgfmathsetmacro{\rotmod}{\tmp>45?90-\tmp:\tmp}
  \pgfmathparse{(0.007*\rotmod-0.315)/45+1.017} \let\rotfudge\pgfmathresult
  \pgfmathparse{1+3.62/(1+(\inverse/0.714)^1.69)} \let\fudge\pgfmathresult
  \pgfmathparse{50*(1-\ratio)*\fudge*\rotfudge} \let\colorstop\pgfmathresult
  \pgfmathparse{(100-\colorstop)} \let\colorstoptwo\pgfmathresult

  \pgfdeclareverticalshading{myshade}{100bp}{%
					color(0bp)=(\ofill);
					color(\colorstop bp)=(\ofill);
					color(50 bp)=(\ifill);
					color(\colorstoptwo bp)=(\ofill);
					color(100bp)=(\ofill)}

  \begin{scope}[rotate around = {\rot:(start)}, rounded corners = \radius*\scale cm, shading angle=\rot]
    \begin{scope} 
      \path[clip]($ (start)+(-0.5*\height, 0.5*\height cm) $) rectangle +(\length pt+\height cm, -\height);
      \shade[shading=myshade] ($ (start)+(-0.5*\height, 0.5*\length pt) $) rectangle +(\length pt+\height cm, -\length pt);
    \end{scope}
  \draw[\stroke, line width = \linewidth mm] ($ (start)+(-0.5*\height, 0.5*\height cm) $) rectangle +(\length pt+\height cm, -\height);
  

  \end{scope}

  
  % \shade[ball color=\ofill] (start) circle (\height/4);
  % \shade[ball color=\ofill] (end) circle (\height/4);

  % \draw(current bounding box.south west) rectangle (current bounding box.north east);


}