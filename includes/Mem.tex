%\Member{startpt}{endpt}{outer fill color}{inner fill color}{stroke}{height}{radius}{linewidth}
\providecommand{\Mem}[8]{
  % name the points
  \coordinate(start) at (#1);
  \coordinate(end) at (#2);
  \edef\ofill{#3}%
  \edef\ifill{#4}%
  \edef\stroke{#5}%
  \edef\height{#6} % cm
  \edef\radius{#7} % cm
  \edef\linewidth{#8} % mm

  \coordinate(delta) at ($ (end)-(start) $);
  \gettikzxy{(delta)}{\dx}{\dy}
  \gettikzxy{(start)}{\sx}{\sy}
  \gettikzxy{(end)}{\ex}{\ey}
  \pgfmathparse{veclen(\dx, \dy)} \let\length\pgfmathresult

  \pgfmathparse{\dx==0}%
  % \ifnum low-level TeX for integers
  \ifnum\pgfmathresult=1 % \dx == 0
    \pgfmathsetmacro{\rot}{\dy > 0 ? 90 : -90}
  \else
    \pgfmathsetmacro{\rot}{\dx > 0 ? atan(\dy / \dx) : 180 + atan(\dy / \dx)}
  \fi
  \pgfmathparse{\dy==0}%
  % \ifnum\pgfmathresult=1 % \dx == 0
  %   \pgfmathsetmacro{\rot}{\dy > 0 ? 0 : 180}
  % \fi

  \pgfdeclareverticalshading{myshading}{\length pt}{
    color(0)=(\ofill); color(0.5*\length pt-0.5*\height cm)=(\ofill); color(0.5*\length pt)=(\ifill); color(0.5*\length pt+0.5*\height cm)=(\ofill); color(\length pt)=(\ofill)
  }

  
  \begin{scope}[shading=myshading] 
    \shadedraw[rotate around = {\rot:(\sx,\sy)}, line width = \linewidth mm, rounded corners = \radius cm, draw = \stroke, shading angle=\rot] ($ (\sx,\sy)+(-0.5*\height cm, 0.5*\height cm) $) -- ++(\height cm +\length pt, 0 ) -- ++(0, -\height cm) -- ++ (-\height cm -\length pt, 0) -- cycle;
    %  \node[below] at (\sx, \sy) {\length};
  \end{scope}

%  \pgfresetboundingbox
  \shadedraw[ball color = \ofill!50!\ifill, draw = \stroke] (start) circle (\height/8);
  \shadedraw[ball color = \ofill!50!\ifill, draw = \stroke] (end) circle (\height/8);
  

  % \node at (\ex,\ey+1cm){$\rot\deg$};
  


}
