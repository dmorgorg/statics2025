\documentclass[10pt,oneside]{article}

\usepackage{amsmath}
\usepackage{bm} 
% \usepackage{textcomp}
\usepackage{mathpazo}
\usepackage{graphicx}
% \usepackage{makebox}
\usepackage[x11names, svgnames]{xcolor} % for \definecolor
\usepackage{tikz}
\usetikzlibrary{calc, arrows.meta}
\usetikzlibrary{decorations.markings, decorations.fractals}
% \usetikzlibrary{decorations.shapes, decorations.text, decorations.pathmorphing, decorations.markings}
\usepackage[letterpaper]{geometry}
\geometry{verbose,tmargin=0.25in,bmargin=0.5in,lmargin=1in,rmargin=1.15in}
\usepackage[absolute,overlay]{textpos}
\setlength{\TPHorizModule}{1.0cm}
\setlength{\TPVertModule}{\TPHorizModule}
\textblockorigin{0.0cm}{0.0cm}  %start all at upper left corner
% not used in this document but macros defines new commands using it so it has to be here
\usepackage[many]{tcolorbox}

% don't place this before \usepackage{xcolor} if using \definecolor
 \definecolor{saitPurple}{RGB}{112,40,119}
 \definecolor{statsMaroon}{rgb}{0.55, 0, 0}
 \definecolor{saitMaroon}{rgb}{0.55, 0, 0}
 \definecolor{saitRed}{RGB}{224,38,37}
 \definecolor{saitBlue}{rgb}{0, 0.59, 0.85}
 \definecolor{statsDeepBlue}{RGB}{0, 99, 167}
 \definecolor{saitDeepBlue}{RGB}{0, 99, 167}
 \definecolor{LightGrey}{RGB}{200,200,200}
%  \definecolor{boxBG}{RGB}{236, 227, 227}
%  \definecolor{boxBG}{RGB}{242, 233, 223}
\usepackage{xcolor}
\usepackage{cancel}
\usepackage{bm}
\usepackage{graphicx}
\usepackage{hyperref}
\usepackage{adjustbox}
\hypersetup{colorlinks, allcolors=.,urlcolor=structure}
\usepackage{booktabs}  % for top and bottom spacing in table cells, \addlinespace
\usepackage[x11names, svgnames]{xcolor} % for colors in handouts, auto loaded in Beamer?
\usepackage{tikz}
\usetikzlibrary{arrows.meta, math, calc, shadows,bending}
\usetikzlibrary{decorations.markings, decorations.fractals, decorations.text} % for chain, etc.
\usetikzlibrary{intersections}
\usepackage{pgfmath}
\usepackage{ifthen}
\usepgfmodule{oo}
\usetikzlibrary{shadings}
% \usetikzlibrary{decorations.shapes}
\usepackage[many]{tcolorbox}
\tcbuselibrary{skins} % for image boxes
\usepackage[absolute,overlay,showboxes]{textpos}
% \usepackage{textpos}
% \textblockorigin{0.0cm}{0.0cm}  %start all at upper left corner
\TPshowboxesfalse

\newcommand\lb{\linebreak}
\newcommand\Ra{\Rightarrow}
\newcommand\cd{\!\cdot\!}
\newcommand\x{\!\times\!}
\newcommand\pars{\par\smallskip}
\newcommand\parm{\par\medskip}
\newcommand\parb{\par\bigskip}
\renewcommand{\deg}{^\circ}

% counter for resuming enumerated list numbers
\newcounter{resumeenumi}
\newcommand{\suspend}{\setcounter{resumeenumi}{\theenumi}}
\newcommand{\resume}{\setcounter{enumi}{\theresumeenumi}}



% https://tex.stackexchange.com/questions/33703/extract-x-y-coordinate-of-an-arbitrary-point-in-tikz
\makeatletter
\providecommand{\gettikzxy}[3]{%
	\tikz@scan@one@point\pgfutil@firstofone#1\relax
	\edef#2{\the\pgf@x}%
	\edef#3{\the\pgf@y}%
}
\makeatother

\makeatletter
\newcommand{\verbatimfont}[1]{\def\verbatim@font{#1}}%
\makeatother

%%%%%%%%%%%%%%%%%%%%%%%%%%%%%%%%%%%%%%%%%%%%%%%%%%%%%%%%%%%%%%%%%%%%%%%%%%%%%%%%


% \newcommand{\tb}[4][0.8]{
% 	\begin{textblock*}{#1}(#2, #3)
% 		\raggedright
% 		#4
% 	\end{textblock*}
% }

% \def\tb

\newtcolorbox{statsbox}[2][] { 
  colback=white,
  colbacktitle=structure,
  colframe=structure,
  coltitle=white,  
  top=0.25cm,
	bottom=0.125cm,
	left=0mm,
	right=0mm,
  % fonttitle=\itshape\rmfamily,
  halign=flush left, 
  enhanced,
  drop fuzzy shadow,
  attach boxed title to top left={xshift=3.5mm, yshift=-2mm},
  title={#2}, #1}
\newtcolorbox{redbox}{colback=white, colframe=structure, enhanced, drop fuzzy shadow}
\newtcolorbox{titledbox}[1]{colback=white,colframe=structure,title={#1}}
\newtcbox{\tcb}[1][]{colback=white,boxsep=0pt,top=0.5cm,bottom=0.5cm,left=0.5cm,
		right=0.5cm, colframe=structure,  enhanced, drop fuzzy shadow, #1}
\newtcbox{\tcbfig}[1][1]{colback=white,boxsep=0pt,top=0.5cm,bottom=0.5cm,left=0.5cm,
		right=0.5cm, colframe=structure,  enhanced, drop fuzzy shadow, #1}
% tcb title
\newtcbox{\tcbt}[2][]{colback=white,boxsep=0pt,top=5pt,bottom=5pt,left=5pt,
		right=5pt, colframe=structure, enhanced, drop fuzzy shadow,  title={#2}, #1}
% tcb left title
\newtcbox{\tcbtl}[2][]{ colback=white,
  colbacktitle=structure,
  colframe=structure,
  coltitle=white,  
  top=0.25cm,
	bottom=0.125cm,
	left=0mm,
	right=0mm,
  % fonttitle=\bfseries,
  halign=flush left, 
  enhanced,
  drop fuzzy shadow,
  attach boxed title to top left={xshift=3.5mm, yshift=-2mm}, 
	title={#2}, #1}

\newtcbtheorem{myexam}{Example}%
{
	enhanced,
	colback=white,
	colframe=structure,
	% fonttitle=\bfseries,
	fonttitle=\itshape\rmfamily,
	drop fuzzy shadow,
	%description font=\mdseries\itshape,
	attach boxed title to top left={yshift=-2mm, xshift=5mm},
	colbacktitle=structure
	}{exam}% then \pageref{exer:theoexample} references the theo

% \newcommand{\myexample}[2][red]{
% 	% \tcb\tcbset{theostyle/.style={colframe=red,colbacktitle=yellow}}
% 	\begin{myexam}{}{}
% 		#2
% 	\end{myexam}
% 	% \tcbset{colframe=structure,colbacktitle=structure}
% }

\newtcbtheorem{myexer}{Exercise}%
{
	enhanced,
	colback=white,
	colframe=structure,
	% fonttitle=\bfseries,
	drop fuzzy shadow,
	fonttitle=\itshape\rmfamily,
	% description font=\mdseries\itshape,
	attach boxed title to top left={yshift=-2mm, xshift=5mm},
	colbacktitle=structure
	}{exer}



\newcommand{\mini}[2][0.8]{
	\begin{minipage}[c]{#1\columnwidth}
		\raggedright
		#2
	\end{minipage}
}
\newcommand{\minit}[2][0.8]{
	\begin{minipage}[t]{#1\columnwidth}
		% \raggedright
		#2
	\end{minipage}
}

% centered minipage with text \raggedright
%\cmini[width]{content}
\newcommand{\cmini}[2][0.8]{
	\begin{center}
		\begin{minipage}{#1\columnwidth}
			\raggedright
			#2
		\end{minipage}
	\end{center}
}

\newcommand{\fig}[2][1]{% scaled graphic
	\includegraphics[scale=#1]{#2}
}

% centred framed box black border
%\cbox[width]{content}
\newcommand{\cbox}[2][1]{% framed centered color box
	\setlength\fboxsep{5mm}
	\setlength\fboxrule{.2 mm}
	\begin{center}
		\fcolorbox{black}{white}{
			\vspace{-0.5cm}
			\begin{minipage}{#1\columnwidth}
				\raggedright
				#2
			\end{minipage}
		}
	\end{center}
	\setlength\fboxsep{0cm}
}

\newcommand{\ccbox}[4][1]{% framed centered color box
	\setlength\fboxsep{5mm}
	\setlength\fboxrule{.2 mm}
	\begin{center}
		\fcolorbox{#2}{#3}{
			% \vspace{-0.5cm}
			\begin{minipage}{#1\columnwidth}
				\vspace{-0.25cm}
				\raggedright				
				#4
				\vspace{-0.325cm}
			\end{minipage}
		}
	\end{center}
	\setlength\fboxsep{0cm}
}

\newcommand{\cfig}[2][1]{% centred, scaled graphic
	\begin{center}
		% \fcolorbox{structure}{white}{
		\tcbincludegraphics{
			\includegraphics[scale=#1]{#2}
		}
	\end{center}
}

% figure with tight border for photos
% \cfigb[saitMaroon]{borderwidth with unit}{scale}{image}
\newcommand{\stcsfig}[2][1]{
	% \usepackage{adjustbox}
	% \setlength{\fboxrule}{1pt}
	\begin{center}
		\tcbincludegraphics[width=#1\textwidth, boxrule=2pt, top=-3pt, right=-3pt, left=-3pt, bottom=-3pt,colframe=structure, sharp corners, enhanced, drop fuzzy shadow]{#2}
	\end{center}
}






%\Member{startpt}{endpt}{outer fill color}{inner fill color}{stroke}{height}{radius}{linewidth}
\providecommand{\Member}[8]{
  % name the points
  \coordinate(start) at (#1);
  \coordinate(end) at (#2);
  \edef\ofill{#3}%
  \edef\ifill{#4}%
  \edef\stroke{#5}%
  \edef\height{#6} % cm
  \edef\radius{#7} % cm
  \edef\linewidth{#8} % mm

  \coordinate(delta) at ($ (end)-(start) $);
  \gettikzxy{(delta)}{\dx}{\dy}
  \gettikzxy{(start)}{\sx}{\sy}
  \pgfmathparse{veclen(\dx, \dy)} \let\length\pgfmathresult

  \pgfmathparse{\dx==0}%
  % \ifnum low-level TeX for integers
  \ifnum\pgfmathresult=1 % \dx == 0
    \pgfmathsetmacro{\rot}{\dy > 0 ? 90 : -90}
  \else
    \pgfmathsetmacro{\rot}{\dx > 0 ? atan(\dy / \dx) : 180 + atan(\dy / \dx)}
  \fi

  
   
  \shadedraw[transform canvas = { rotate around = {\rot:(\sx,\sy)}}, line width = \linewidth, rounded corners = \radius mm, top color = \ofill, bottom color = \ofill, middle color = \ifill, draw = \stroke] ($ (start)+(-0.5*\height, 0.5*\height) $) -- ++(\height cm +\length pt, 0 ) -- ++(0, -\height) -- ++ (-\height cm -\length pt, 0) -- cycle;


  \shadedraw[ball color = \ofill!50!\ifill, draw = \stroke] (start) circle (\height/8);
  \shadedraw[ball color = \ofill!50!\ifill, draw = \stroke] (end) circle (\height/8);
  %  \pgfresetboundingbox

  
  


}


\newcommand{\PC}[6][0]{%
  \edef\lrotate{#1}%
  \edef\lpin{#2}%
  \edef\lfill{#3}%
  \edef\ldraw{#4}%
  \edef\lscale{#5}%
  \edef\lwidth{#6}%
  \edef\h{1}%
  \edef\r{0.3}%
  \begin{scope}[scale=\lscale, rotate=\lrotate]
	\filldraw[draw=\ldraw, fill=\lfill, line width=\lwidth mm] ($ (\lpin) + (0.201*\h+1.0353*\r ,-0.75*\h) $) -- ++(105: 0.77646*\h+0.26795*\r) arc (15:165:\r) -- ++(-105:0.77646*\h+0.26795*\r) -- cycle;

	\shadedraw[ball color=\lfill, draw=\ldraw, line width = \lwidth mm] (\lpin) circle (1.5mm);

	\filldraw[rounded corners=\lscale pt, draw=\ldraw, fill=\lfill, line width=\lwidth mm] ($ (\lpin) - (1,1) $) rectangle +(2,0.25);
  \end{scope}%
}



% !TEX root = ../../Beamer/statikz/statikz.tex


\newcommand{\EyeConnection}[6][0]{
	\def\lrotate{#1};
	\def\lpin{#2}
	\def\lfill{#3}
	\def\ldraw{#4}
	\def\lscale{#5}
	\def\lwidth{#6}
	\def\h{1}
	\def\r{0.3}
	\begin{scope}[scale=\lscale, rotate=\lrotate]
		\filldraw[draw=\ldraw, fill=\lfill, line width=\lwidth pt] ($(\lpin) + (0.201*\h+1.0353*\r ,-0.75*\h)$) -- ++(105: 0.77646*\h+0.26795*\r) arc (15:165:\r) -- ++(-105:0.77646*\h+0.26795*\r) -- cycle;

		\fill[outer color=\lfill, middle color=red, inner color=black, line width = \lwidth pt] (\lpin) circle (2.5mm);
		\filldraw[fill=white, draw=\ldraw, line width = \lwidth pt] (\lpin) circle (1.25mm);

		\filldraw[rounded corners=\lscale pt, draw=\ldraw, fill=\lfill, line width=\lwidth pt] ($ (\lpin) - (1,1) $) rectangle +(2,0.25);
	\end{scope}
}


% https://tex.stackexchange.com/questions/731957/how-to-supress-missing-character-there-is-no-u003b-in-font-nullfont
\tracinglostchars=1

\hfuzz=150pt
\setlength{\parindent}{0pt}

\begin{document}

%%%%%%%%%%%%%%%%%%%%%%%%%%%%%%%%%%%%%%%%%%%%%%%%%%%%%%%%%%%%%%%%%%%%%%%%%%%%%%%%%%%%%%%%%%%%%%%%%%%%
% page 1
%%%%%%%%%%%%%%%%%%%%%%%%%%%%%%%%%%%%%%%%%%%%%%%%%%%%%%%%%%%%%%%%%%%%%%%%%%%%%%%%%%%%%%%%%%%%%%%%%%%%

\begin{textblock*}{7.25in}(1in, 0.4in)
	\begin{tikzpicture}[line width=0.1mm]
		\draw[color=gray!50, step=0.25in] (0,0) grid +(7.25in,10.25in);
	\end{tikzpicture}
\end{textblock*}


\begin{textblock*}{6.775in}(1in, 0.225in)
  \cbox{
    \centering\huge
    \textbf{Engineering Statics - 02 Force Vectors Handout}
  }
\end{textblock*}

\begin{textblock*}{4in}(1in, 1in)
	\cbox{
		\underline{Example 1} \parm
			      % A displacement is a change in position. It has a magnitude (the distance moved) and a direction, so displacement is a vector quantity.\parb
			      A truck drives due east on a straight road for 40 km, then drives north on a straight road for 30 km before stopping. \parb
			      What is the resultant displacement of the truck?

	}
\end{textblock*}

\begin{textblock*}{4in}(1in, 5in)
	\cbox{
		\underline{Example 2} \parm
		A plane flies NNW (i.e., $22.5\deg$ west of north) with a velocity of $275\text{ km/h}$. There is a wind blowing at $55\text{ km/h}$ from the NW \lb(i.e., $45\deg$ west of north).
			      \parb Determine the resultant velocity of the plane relative to the ground.
			      \parb Determine the wind speed that would cause the plane to fly due north. What is the ground speed in this case?

	}
\end{textblock*}
%%%%%%%%%%%%%%%%%%%%%%%%%%%%%%%%%%%%%%%%%%%%%%%%%%%%%%%%%%%%%%%%%%%%%%%%%%%%%%%%%%%%%%%%%%%%%%%%%%%%
% page 2
%%%%%%%%%%%%%%%%%%%%%%%%%%%%%%%%%%%%%%%%%%%%%%%%%%%%%%%%%%%%%%%%%%%%%%%%%%%%%%%%%%%%%%%%%%%%%%%%%%%%

.\newpage
\begin{textblock*}{7.25in}(1in, 0.4in)
	\begin{tikzpicture}[line width=0.1mm]
		\draw[color=gray!50, step=0.25in] (0,0) grid +(7.25in,10.25in);
	\end{tikzpicture}
\end{textblock*}

\begin{textblock*}{3.5in}(1in, 0.225in)
	\cbox{
		\underline{Example 3} \parm
		Determine the magnitude and the direction (measured clockwise from the the positive $x$-axis) of the resultant of the two forces.

	}
\end{textblock*}
\begin{textblock*}{2.75in}(5.03in, 0.225in)
	\def\scale{0.95}
	\cbox{
		\centering
		
		\begin{tikzpicture}[scale=\scale, line cap=round,
			postaction=decorate,
  		decoration = {markings, mark = between positions 0
  		and 1 step 16pt with 
				{
					\begin{scope}[scale=0.8]
						\draw[gray, very thick]  (0, -3pt)--++(3pt, 0)arc(-90:90:3pt)--++(-6pt,0)arc(90:270:3pt)--cycle;
						\draw[gray,ultra thick] (-16pt,0) -- (-4pt,0); 
					\end{scope}
				}
			}
		]

	\coordinate (D) at (-3,1);
	\coordinate (E) at (3, 2);
	\coordinate (C) at (0, 0);

	\fill[gray!50] (D) rectangle (E);
	\draw[thin, black] (D) -- ($(E)-(0,1) $);
	\EyeConnection[180]{C}{gray!50}{black}{1}{0.5};

	\draw[latex-latex] ($ (C)+(180:2)$) arc (180:225:2) node[fill=white, midway, inner sep=0.25mm] {$ 45\deg $};
	\draw[latex-latex] ($ (C)+(0:2)$) arc (0:-30:2) node[fill=white, midway, inner sep=0.25mm] {$ 30\deg $};


	\path[postaction={decorate}] ($ (C)+(-30:0.55) $) -- +(-30:2);
	\path[postaction={decorate}] ($ (C)+(225:0.55) $) -- +(225:2);
	\draw[line width=0.875mm, -latex] ($ (C)+(-30:1.25) $) -- +(-30:2) node[black, below left]{\(2.50\,\text{ kN}\)};
	\draw[line width=0.875mm, -latex] ($ (C)+(225:1.25) $) -- +(225:2) node[black, below]{\(3.75\,\text{ kN}\)};

	\draw[thin] ($(C)+(-0.5,0) $) -- +(-2.25,0);
	\draw[thin] ($(C)+(0.5,0) $) -- +(2.25,0) node[right] {$x$};



	


\end{tikzpicture}
	}
\end{textblock*}

\begin{textblock*}{3in}(1in, 5in)
	\cbox{
		\underline{Exercise 1} \parm
		The resultant of the forces $F$ and $F_1$ is $3.14$ kN at $37^\circ$ clockwise from the
					positive $x$ axis.\parm
					Determine $F$ and $\theta$.

	}
\end{textblock*}
\begin{textblock*}{3.25in}(4.525in, 5in)
	\cbox{
		\centering
		\def\scale{0.8}
		\begin{tikzpicture}[scale=0.8, line cap=round,
			postaction=decorate,
  		decoration = {markings, mark = between positions 0
  		and 1 step 16pt with 
				{
					\begin{scope}[scale=0.8]
						\draw[gray, very thick]  (0, -3pt)--++(3pt, 0)arc(-90:90:3pt)--++(-6pt,0)arc(90:270:3pt)--cycle;
						\draw[gray,ultra thick] (-16pt,0) -- (-4pt,0); 
					\end{scope}
				}
			}
		]

	\coordinate (D) at (-3,1);
	\coordinate (E) at (3, 2.5);
	\coordinate (C) at (0, 0);

\begin{scope}[rotate=25]
	\fill[gray!50] (-3,1) rectangle (3,2.5);
	\draw[thin, black] (-3,1) -- ($(3,2.5)-(0,1.5)$);
	\EyeConnection[180]{C}{gray!50}{black}{1}{0.5}
\end{scope}

	\draw[latex-latex] ($(C)+(180:1.65)$) arc (180:237:1.65) node[fill=white, midway, inner sep=0.25mm] {$\bm \theta$};
	\draw[latex-latex] ($(C)+(0:2.45)$) arc (0:25:2.45) node[fill=white, midway, inner sep=0.25mm] {$ 26^\circ $};


	\path[postaction={decorate}] ($(C)+(25:0.625)$) -- +(25:2);
	\path[postaction={decorate}] ($(C)+(237:0.625)$) -- +(237:2);
	\draw[line width=0.875mm, -latex] ($(C)+(25:1.375)$) -- +(25:2) node[black, right]{$F_1=4.72\,\mathrm{kN}$};
	\draw[line width=0.875mm, -latex] ($(C)+(237:1.375)$) -- +(237:2) node[left]{\Large $\bm F $};

	\draw[thin] ($(C)+(-0.5,0)$) -- +(-1.5,0);
	\draw[thin] ($(C)+(0.5,0)$) -- +(2.25,0) node[right] {$x$};




\end{tikzpicture}

	}
\end{textblock*}



%%%%%%%%%%%%%%%%%%%%%%%%%%%%%%%%%%%%%%%%%%%%%%%%%%%%%%%%%%%%%%%%%%%%%%%%%%%%%%%%%%%%%%%%%%%%%%%%%%%%
% page 3
%%%%%%%%%%%%%%%%%%%%%%%%%%%%%%%%%%%%%%%%%%%%%%%%%%%%%%%%%%%%%%%%%%%%%%%%%%%%%%%%%%%%%%%%%%%%%%%%%%%%
.\newpage
\begin{textblock*}{7.25in}(1in, 0.4in)
  % \textblockcolor{pink}
	\begin{tikzpicture}[line width=0.1mm]
		\draw[color=gray!50, step=0.25in] (0,0) grid +(7.25in,10.25in);
	\end{tikzpicture}
\end{textblock*}

\begin{textblock*}{4in}(1in, 0in)
	\cbox{
		\underline{Example 4} \parm
		The weight, $W$, of the traffic lights (with mass $105\;\text{kg}$) acts vertically downward.
					\parm
					Find the value of $ W$ and use it to determine the magnitudes of its two components directed along the axes of $AC$ and $BC$.

	}
\end{textblock*}
\begin{textblock*}{2.25in}(5.525in, 0in)
	\cbox{
		\centering
		\scalebox{1.25}{
\begin{tikzpicture}[scale=0.5]

	\coordinate (C) at (0,0);
	\coordinate (B) at ($ (C)+(-15:5.176) $);
	\coordinate (A) at ($ (C)+(-55:8.717) $);
	\coordinate (D) at ($ (C)+(0,-3) $);

	\gettikzxy{(A)}{\ax}{\ay}
  \gettikzxy{(B)}{\bx}{\by}
  \gettikzxy{(C)}{\cx}{\cy}

	\PC[90]{B}{Seashell4}{black}{0.67}{0.125}
	\PC[90]{A}{Seashell4}{black}{0.67}{0.125}

	\fill[gray] ($ (\bx+0.67cm, \cy+1cm) $) rectangle ($ (\ax+2cm, \ay-2cm) $);
	\draw[thin]  ($ (\bx+0.67cm, \cy+1cm) $) -- ($ (\bx+0.67cm, \ay-2cm) $);
	
	\Member{C}{A}{gray!75!white}{gray!25!white}{black}{0.5}{1.125}{0.5}	
	\Member{C}{D}{gray!75!white}{gray!25!white}{black}{0.2}{.125}{0.5}
	\Member{C}{B}{gray!75!white}{gray!25!white}{black}{0.5}{1.125}{0.5}

	\filldraw[rounded corners, fill=Seashell4] ($ (D)+(-0.5,-1.5) $) rectangle ($ (D)+(0.5,1.5) $);
	\shadedraw[ball color=gray!60!black] (D) circle (.3cm);
	\shadedraw[ball color=gray!60!black] ($(D)+(0,0.75)$) circle (.3cm);
	\shadedraw[ball color=gray!20] ($(D)+(0,-0.75)$) circle (.3cm);
	\node at ($ (D)-(-0.5,2) $) {\footnotesize $ 105\,$kg};

	\draw[Latex-Latex] ($ (B)+(-15:0.6936)+(0,1.25) $) arc (90:165:1.25) node[midway,xshift=-1.5mm, yshift=1.3mm] {\footnotesize $75\deg$};
	\draw[Latex-Latex] ($ (A)+(-55:1.168)+(0,2.4) $) arc (90:125:2.4) node[midway,yshift=1.75mm] {\footnotesize $35\deg$};

	\small
	
	\shadedraw [draw=black] (B) circle (0.1cm) node[xshift=-0.2cm, yshift=-0.25cm] {$B$};
	\shadedraw [draw=black] (A) circle (0.1cm) node[xshift=-0.2cm, yshift=-0.2cm] {$A$};
	\shadedraw [draw=black] (C) circle (0.1cm) node[xshift=-0.3cm, yshift=-0.1cm] {$C$};

	\pgfresetboundingbox
	\draw[white] (\cx-1.5cm, \cy+1cm) rectangle (\ax+2cm, \ay-2cm);
	

\end{tikzpicture}


}
		
	}
\end{textblock*}

\begin{textblock*}{2.5in}(1in, 5in)
	\cbox{
	\underline{Exercise 2} \parm
		Resolve the $4.20$ kN load suspended from $A$ into components parallel to the truss members $AB$ and $AG$.
		\parm
		Give the magnitude of the components and their direction measured counter-clockwise from the positive $x$ axis.

	}
\end{textblock*}
\begin{textblock*}{3.75in}(4.025in, 5in)
	\cbox{
		\centering
		\scalebox{0.8}{
			\tikz{%
  \coordinate (A) at (0,0);
  \coordinate (B) at (2.5,0);
  \coordinate (C) at (5.5,0);
  \coordinate (D) at (8,1.25);
  \coordinate (E) at (8,-3);
  \coordinate (F) at (5.5,-3);
  \coordinate (G) at (2.5,-3);

  \gettikzxy{(C)}{\Cx}{\Cy};
	\gettikzxy{(E)}{\Ex}{\Ey};
	\gettikzxy{(D)}{\Dx}{\Dy};

  \fill[gray] ($ (E)+(0.5,-1) $) rectangle ($ (D)+(1.5,1.5) $);
	\draw[black, thick] ($ (E)+(0.5,-1) $) -- ($ (D)+(0.5,1.5) $);

  \PC[90]{D}{gray}{black}{0.5}{0.25}
	\PC[90]{E}{gray}{black}{0.5}{0.25}

  \Member{A}{B}{gray}{white}{black}{0.3125}{1.5}{0.25}
  \Member{B}{C}{gray}{white}{black}{0.3125}{1.5}{0.25}
  \Member{C}{D}{gray}{white}{black}{0.3125}{1.5}{0.25}
  \Member{E}{F}{gray}{white}{black}{0.3125}{1.5}{0.25}
  \Member{F}{G}{gray}{white}{black}{0.3125}{1.5}{0.25}
  \Member{A}{G}{gray}{white}{black}{0.3125}{1.5}{0.25}
  \Member{C}{G}{gray}{white}{black}{0.3125}{1.5}{0.25}
  \Member{C}{E}{gray}{white}{black}{0.3125}{1.5}{0.25}
  \Member{B}{G}{gray}{white}{black}{0.3125}{1.5}{0.25}
  \Member{C}{F}{gray}{white}{black}{0.3125}{1.5}{0.25}

  \draw[thin, black] ($ (A)+(0,0.325)$) -- +(0,2.25);
  \draw[thin, black] ($ (B)+(0,0.325)$) -- +(0,2.25);
  \draw[thin, black] ($ (C)+(0,0.325)$) -- +(0,2.25);
  \draw[thin, black] ($ (D)+(0,0.325)$) -- +(0,1);
  \draw[thin, black] ($ (D)+(0.25,0)$) -- +(2.25,0);
  \draw[thin, black] ($ (E)+(0.25,0)$) -- +(2.25,0);
  \draw[thin, black] ($ (C)+(0.5,0)$) -- +(4.5,0);

  \draw[thick, latex-latex] ([yshift=2.125cm]A) -- node[fill=white, inner sep=0.375mm]{$2.50$ m}([yshift=2.125cm]B);
	\draw[thick, latex-latex] ([yshift=2.125cm]B) -- node[fill=white, inner sep=0.375mm]{$3.00$ m}([yshift=2.125cm]C);
	\draw[thick, latex-latex] ([yshift=2.125cm]C) -- node[fill=white, inner sep=0.375mm]{$2.50$ m}([yshift=0.875cm]D);
	\draw[thick, latex-latex] (10.125, \Dy) -- node[fill=white, inner sep=0.75mm]{$1.25$ m}(10.125, \Cy);
	\draw[thick, latex-latex] (10.125, \Cy) -- node[fill=white, inner sep=0.75mm]{$3.00$ m}(10.125, \Ey);


  \draw[line width=0.625mm, -Latex, line cap=round] (A) -- +(0,-2.5) node[below] {\Large {$\bm{4.20}$ \bf kN}};

  \fill[ball color=gray] (A) circle (3pt) node[above left, inner sep=2mm] {\large $\bm A$};
	\fill[ball color=gray] (B) circle (3pt) node[above left, inner sep=2mm] {\large $\bm B$};
	\fill[ball color=gray] (C) circle (3pt) node[above left, inner sep=2mm] {\large $\bm C$};
	\fill[ball color=gray] (D) circle (3pt) node[above left, inner sep=1.5mm] {\large $\bm D$};
	\fill[ball color=gray] (E) circle (3pt) node[below left, inner sep=2.5mm] {\large $\bm E$};
	\fill[ball color=gray] (F) circle (3pt) node[below left, inner sep=2mm] {\large $\bm F$};
	\fill[ball color=gray] (G) circle (3pt) node[below left, inner sep=2mm] {\large $\bm G$};


  % \pgfresetboundingbox
  \useasboundingbox ($ (A)+ (-0.75,3) $) rectangle ($ (E)+(2.875,-1.25) $);





}
		}
	}
\end{textblock*}




%%%%%%%%%%%%%%%%%%%%%%%%%%%%%%%%%%%%%%%%%%%%%%%%%%%%%%%%%%%%%%%%%%%%%%%%%%%%%%%%%%%%%%%%%%%%%%%%%%%%
% page 4
%%%%%%%%%%%%%%%%%%%%%%%%%%%%%%%%%%%%%%%%%%%%%%%%%%%%%%%%%%%%%%%%%%%%%%%%%%%%%%%%%%%%%%%%%%%%%%%%%%%%
.\newpage
\begin{textblock*}{7.25in}(1in, 0.375in)
	\begin{tikzpicture}[line width=0.1mm]
		\draw[color=gray!50, step=0.25in] (0,0) grid +(7.25in,10.25in);
	\end{tikzpicture}
\end{textblock*}

\begin{textblock*}{3.75in}(1in, 0.235in)
	\cbox{
		\underline{Exercise 3} \parm
		The decoration suspended at $D$ weighs $1124\,\mathsf{N}$.\parm 
				Determine the magnitudes of the two force components of the weight of $D$, in the direction of $AB$ and $BC$.

	}
\end{textblock*}
\begin{textblock*}{2.5in}(5.275in, 0.235in)
	\cbox{
		\centering
		\def\scale{0.8}
		
\begin{tikzpicture}[scale=\scale, decoration=Koch snowflake,fill=gray!50,thick]

	\coordinate (A) at (0,-2);
	\coordinate (AA) at ($(A)+(-26:1)$);
	\coordinate (C) at (5,-1);
	\coordinate (CC) at ($(C)+(214:1)$);
	\coordinate (B) at (intersection of A--AA and C--CC);
	\coordinate (D) at ($ (B)+(0,-1.6875) $);

	\fill[gray!50] ($ (A)+(0,2) $) rectangle  ($ (A)+(-1,-4.5) $); 
	\fill[gray!50] ($ (C)+(0,1) $) rectangle  ($ (C)+(1,-5.5) $); 
	\draw[thick] ($ (A)+(0,2) $) -- ($ (A)+(0,-4.5) $); 
	\draw[thick] ($ (C)+(0,1) $) -- ($ (C)+(0,-5.5) $); 


	\draw[very thick] (A)--(B);
	\draw[very thick] (B)--(C);
	\draw[very thick] (B)--(D);

	\filldraw[xshift=1.5375cm, yshift=-2.5cm] decorate{ decorate{ (0,-2.5) -- ++(60:1) -- ++(-60:1) -- cycle }};


	\draw[thin, latex-latex] ($ (B)+(154:1.25) $) arc (154:34:1.25) node[fill=white, inner sep=0.25mm, midway] {\footnotesize $120\deg$};
	\draw[thin, latex-latex] ($ (C)+(270:1.75) $) arc (270:214:1.75) node[fill=white, inner sep=0.25mm, midway] {\footnotesize $56.0\deg$};


	\shadedraw [draw=black, ball color = gray] (B) circle (0.1cm) node[above] {$B$};
	\shadedraw [draw=black, ball color = gray] (A) circle (0.1cm) node[left] {$A$};
	\shadedraw [draw=black, ball color = gray] (C) circle (0.1cm) node[right] {$C$};
	\node at (D) {$\bm D$};
	\node[black, xshift=0.5cm, yshift=-0.625cm] at (D) {\footnotesize $1124\,\mathsf{N}$};



\end{tikzpicture}

	}
\end{textblock*}

% \begin{textblock*}{3in}(1in, 5.25in)
% 	\cbox{
% 	Example 5 \parm
% 	\begin{enumerate}[label=\alph*)]
% 		\item Determine the resultant ${\bm{\mathrm{ R }}}$ of the two vectors ${\bm{\mathrm{ F }}}$ and ${\bm{\mathrm{ G }}}$.
% 		\item Determine the $x$-component of ${\bm{\mathrm{ R }}}$ \lb(i.e., the horizontal component).
% 		% \item Determine the $y$-component of ${\bm{\mathrm{ R }}}$ \lb(i.e., the vertical component).
% 		\item Determine the $x$-component of ${\bm{\mathrm{ F }}}$.
% 		\item Determine the $x$-component of ${\bm{\mathrm{ G }}}$.
% 		\item Add the two previous results.
% 	\end{enumerate}

% 	}
% \end{textblock*}
\begin{textblock*}{3.25in}(4.625in, 5.25in)
	\cbox{
		% \cfig[0.125]{../../Figs/02ConcurrentForces/conc2016A}
	}
\end{textblock*}


%%%%%%%%%%%%%%%%%%%%%%%%%%%%%%%%%%%%%%%%%%%%%%%%%%%%%%%%%%%%%%%%%%%%%%%%%%%%%%%%%%%%%%%%%%%%%%%%%%%%
% page 5
%%%%%%%%%%%%%%%%%%%%%%%%%%%%%%%%%%%%%%%%%%%%%%%%%%%%%%%%%%%%%%%%%%%%%%%%%%%%%%%%%%%%%%%%%%%%%%%%%%%%
.\newpage
\begin{textblock*}{7.25in}(1in, 0.375in)
	\begin{tikzpicture}[line width=0.1mm]
		\draw[color=gray!50, step=0.25in] (0,0) grid +(7.25in,10.25in);
	\end{tikzpicture}
\end{textblock*}


%%%%%%%%%%%%%%%%%%%%%%%%%%%%%%%%%%%%%%%%%%%%%%%%%%%%%%%%%%%%%%%%%%%%%%%%%%%%%%%%%%%%%%%%%%%%%%%%%%%%
% page 6
%%%%%%%%%%%%%%%%%%%%%%%%%%%%%%%%%%%%%%%%%%%%%%%%%%%%%%%%%%%%%%%%%%%%%%%%%%%%%%%%%%%%%%%%%%%%%%%%%%%%
.\newpage
\begin{textblock*}{7.25in}(1in, 0.375in)
	\begin{tikzpicture}[line width=0.1mm]
		\draw[color=gray!50, step=0.25in] (0,0) grid +(7.25in,10.25in);
	\end{tikzpicture}
\end{textblock*}


%%%%%%%%%%%%%%%%%%%%%%%%%%%%%%%%%%%%%%%%%%%%%%%%%%%%%%%%%%%%%%%%%%%%%%%%%%%%%%%%%%%%%%%%%%%%%%%%%%%%
% page 7
%%%%%%%%%%%%%%%%%%%%%%%%%%%%%%%%%%%%%%%%%%%%%%%%%%%%%%%%%%%%%%%%%%%%%%%%%%%%%%%%%%%%%%%%%%%%%%%%%%%%
.\newpage
\begin{textblock*}{7.25in}(1in, 0.375in)
	\begin{tikzpicture}[line width=0.1mm]
		\draw[color=gray!50, step=0.25in] (0,0) grid +(7.25in,10.25in);
	\end{tikzpicture}
\end{textblock*}



\end{document}