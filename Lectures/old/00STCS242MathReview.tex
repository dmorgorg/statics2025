%!TEX option = --enable-write18


\documentclass[9pt,xcolor=svgnames,professionalfonts, mathserif]{beamer}
\usepackage{amsmath}
\usepackage{amssymb}
\usepackage{graphicx}
\usepackage{booktabs}  % for top and bottom spacing in table cells
\usepackage{mathpazo}
\usepackage{textcomp}
\usepackage{gensymb}
\usepackage{multirow}
\usepackage{cancel}
\usepackage{array}
%\usepackage{enumerate}
% \usepackage{enumitem} %causes compile error, stack size exceeded?
\usepackage[many]{tcolorbox}
\usepackage{verbatim}
\usepackage{bm}
\usepackage{graphicx}
\usepackage{tikz}
\usepackage{tkz-linknodes}
\usetikzlibrary{shapes,decorations,shadows}
\usetikzlibrary{decorations.shapes}
\usetikzlibrary{shapes.callouts}

\usefonttheme{structureitalicserif} %make titles fancy ;-)

\usepackage[absolute,overlay]{textpos}
\setlength{\TPHorizModule}{1.0cm}
\setlength{\TPVertModule}{\TPHorizModule}
\textblockorigin{0.0cm}{0.0cm}  %start all at upper left corner
\usepackage{hyperref}
\hypersetup{colorlinks=true, urlcolor=Maroon}
% \hypersetup{urlcolor=Blue4}

\setlength{\parskip}{\medskipamount}
\setlength{\parindent}{0pt}

\usetheme{Antibes}

\usecolortheme[named=Maroon]{structure}
\definecolor{structurecolor}{rgb}{0.55,0.53,0.31}
\setbeamertemplate{items}[triangle]
\setbeamertemplate{blocks}[rounded][shadow=false]
%\setbeamertemplate{background canvas}[vertical shading][bottom=Cyan1!50, middle=white, top=white, midpoint=0.05]
\setbeamertemplate{headline}{\vspace{.1cm}}
\setbeamertemplate{footline}{\hfill \insertshorttitle \quad \insertshortsubtitle \quad \insertframenumber/\inserttotalframenumber \quad{ }\vspace{0.125cm}}
\addtobeamertemplate{footline}{\hypersetup{linkcolor=.}}{}
\setbeamertemplate{navigation symbols}{} % empty braces suppresses all navigation symbols
\setbeamercolor{frametitle}{fg=Linen}
% \setbeamercolor{footline}{fg=black}
\setbeamercolor{block title}{fg=Linen,bg=structure}
\setbeamercolor{block body}{bg=Maroon!25, fg=black}
\setbeamercolor{background canvas}{bg=Linen}
\setbeamersize{text margin left = 1cm, text margin right=1cm}
%\useinnertheme[shadow]{rounded}
\raggedright

\everymath{\displaystyle}

\usepackage{xcolor}
\usepackage{cancel}
\usepackage{bm}
\usepackage{graphicx}
\usepackage[x11names, svgnames]{xcolor} % for colors in handouts, auto loaded in Beamer?
\usepackage{tikz}
\usetikzlibrary{arrows.meta, math, calc, shadows}
\usetikzlibrary{decorations.markings, decorations.fractals, decorations.text} % for chain, etc.
\usetikzlibrary{intersections}
\usepackage{pgfmath}
\usepackage{ifthen}
\usepgfmodule{oo}
\usepgflibrary{shadings}
% \usetikzlibrary{decorations.shapes}
\usepackage[many]{tcolorbox}
\usepackage[absolute,overlay,showboxes]{textpos}
% \usepackage{textpos}
% \textblockorigin{0.0cm}{0.0cm}  %start all at upper left corner
\TPshowboxesfalse

\newcommand\lb{\linebreak}
\newcommand\Ra{\Rightarrow}
\newcommand\cd{\!\cdot\!}
\newcommand\x{\!\times\!}
\newcommand\pars{\par\smallskip}
\newcommand\parm{\par\medskip}
\newcommand\parb{\par\bigskip}
\renewcommand{\deg}{^\circ}

% counter for resuming enumerated list numbers
\newcounter{resumeenumi}
\newcommand{\suspend}{\setcounter{resumeenumi}{\theenumi}}
\newcommand{\resume}{\setcounter{enumi}{\theresumeenumi}}



% https://tex.stackexchange.com/questions/33703/extract-x-y-coordinate-of-an-arbitrary-point-in-tikz
\makeatletter
\providecommand{\gettikzxy}[3]{%
	\tikz@scan@one@point\pgfutil@firstofone#1\relax
	\edef#2{\the\pgf@x}%
	\edef#3{\the\pgf@y}%
}
\makeatother

\makeatletter
\newcommand{\verbatimfont}[1]{\def\verbatim@font{#1}}%
\makeatother

%%%%%%%%%%%%%%%%%%%%%%%%%%%%%%%%%%%%%%%%%%%%%%%%%%%%%%%%%%%%%%%%%%%%%%%%%%%%%%%%


\newcommand{\tb}[4][0.8]{
	\begin{textblock*}{#1}(#2, #3)
		% \raggedright
		#4
	\end{textblock*}
}

\newtcolorbox{statsbox}[2][] { 
  colback=white,
  colbacktitle=structure,
  colframe=structure,
  coltitle=white,  
  top=0.25cm,
	bottom=0.125cm,
	left=0mm,
	right=0mm,
  % fonttitle=\itshape\rmfamily,
  halign=flush left, 
  enhanced,
  drop fuzzy shadow,
  attach boxed title to top left={xshift=3.5mm, yshift=-2mm},
  title={#2}, #1}
\newtcolorbox{redbox}{colback=white, colframe=structure, enhanced, drop fuzzy shadow}
\newtcolorbox{titledbox}[1]{colback=white,colframe=structure,title={#1}}
\newtcbox{\tcb}[1][]{colback=white,boxsep=0pt,top=5pt,bottom=5pt,left=5pt,
		right=5pt, colframe=structure,  enhanced, drop fuzzy shadow, #1}
% tcb title
\newtcbox{\tcbt}[2][]{colback=white,boxsep=0pt,top=5pt,bottom=5pt,left=5pt,
		right=5pt, colframe=structure, enhanced, drop fuzzy shadow,  title={#2}, #1}
% tcb left title
\newtcbox{\tcbtl}[2][]{ colback=white,
  colbacktitle=structure,
  colframe=structure,
  coltitle=white,  
  top=0.25cm,
	bottom=0.125cm,
	left=0mm,
	right=0mm,
  % fonttitle=\bfseries,
  halign=flush left, 
  enhanced,
  drop fuzzy shadow,
  attach boxed title to top left={xshift=3.5mm, yshift=-2mm}, 
	title={#2}, #1}

\newtcbtheorem{myexam}{Example}%
{
	enhanced,
	colback=white,
	colframe=structure,
	% fonttitle=\bfseries,
	fonttitle=\itshape\rmfamily,
	drop fuzzy shadow,
	%description font=\mdseries\itshape,
	attach boxed title to top left={yshift=-2mm, xshift=5mm},
	colbacktitle=structure
	}{exam}% then \pageref{exer:theoexample} references the theo

% \newcommand{\myexample}[2][red]{
% 	% \tcb\tcbset{theostyle/.style={colframe=red,colbacktitle=yellow}}
% 	\begin{myexam}{}{}
% 		#2
% 	\end{myexam}
% 	% \tcbset{colframe=structure,colbacktitle=structure}
% }

\newtcbtheorem{myexer}{Exercise}%
{
	enhanced,
	colback=white,
	colframe=structure,
	% fonttitle=\bfseries,
	drop fuzzy shadow,
	fonttitle=\itshape\rmfamily,
	% description font=\mdseries\itshape,
	attach boxed title to top left={yshift=-2mm, xshift=5mm},
	colbacktitle=structure
	}{exer}



\newcommand{\mini}[2][0.8]{
	\begin{minipage}[c]{#1\columnwidth}
		\raggedright
		#2
	\end{minipage}
}
\newcommand{\minit}[2][0.8]{
	\begin{minipage}[t]{#1\columnwidth}
		% \raggedright
		#2
	\end{minipage}
}

% centered minipage with text \raggedright
%\cmini[width]{content}
\newcommand{\cmini}[2][0.8]{
	\begin{center}
		\begin{minipage}{#1\columnwidth}
			\raggedright
			#2
		\end{minipage}
	\end{center}
}



\newcommand{\fig}[2][1]{% scaled graphic
	\includegraphics[scale=#1]{#2}
}

% centred framed colored box black border
%\cbox[width]{content}
\newcommand{\cbox}[2][1]{% framed centered color box
	\setlength\fboxsep{5mm}
	\setlength\fboxrule{.2 mm}
	\begin{center}
		\fcolorbox{black}{white}{
			\vspace{-0.5cm}
			\begin{minipage}{#1\columnwidth}
				\raggedright
				#2
			\end{minipage}
		}
	\end{center}
	\setlength\fboxsep{0cm}
}

\newcommand{\cfig}[2][1]{% centred, scaled graphic
	\begin{center}
		\includegraphics[scale=#1]{#2}
	\end{center}
}






 \definecolor{saitPurple}{RGB}{112,40,119}
 \definecolor{statsMaroon}{rgb}{0.55, 0, 0}
 \definecolor{saitMaroon}{rgb}{0.55, 0, 0}
 \definecolor{statsRed}{RGB}{224,38,37}
 \definecolor{saitRed}{RGB}{224,38,37}
 \definecolor{saitBlue}{rgb}{0, 0.59, 0.85}
 \definecolor{statsBlue}{rgb}{0, 0.59, 0.85}
 \definecolor{statsDeepBlue}{RGB}{0, 99, 167}
 \definecolor{saitDeepBlue}{RGB}{0, 99, 167}
 \definecolor{saitDeepBlue}{RGB}{0, 99, 167}
 \definecolor{LightGrey}{RGB}{200,200,200}
%  \definecolor{boxBG}{RGB}{236, 227, 227}
%  \definecolor{boxBG}{RGB}{242, 233, 223}

\newcounter{myexercisecounter}


%%%%%%%%%%%%%%%%%%%%%%%%%%%%%%%%%%%%%%%%%%%%%%%%%%%%%%%%%%%%%%%%%%%%%%%%%%%%%%%%
%define title content
\title[Math Review]{\Huge \textcolor{white}{Math Review}}
\subtitle[STCS242]{\Large\textcolor{white}{Structural Statics, STCS 242}}
\author{Dave Morgan}
\institute{\large SAIT Polytechnic}
\date{\small \textcolor{gray}{Last revision on \today}}

%%%%%%%%%%%%%%%%%%%%%%%%%%%%%%%%%%%%%%%%%%%%%%%%%%%%%%%%%%%%%%%%%%%%%%%%%%%%%%%%

\begin{document}


%%%%%%%%%%%%%%%%%%%%%%%%%%%%%%%%%%%%%%%%%%%%%%%%%%%%%%%%%%%%%%%%%%%%%%%%%%%%%%%%

\begin{frame}[plain]    %don't need footer on titlepage
	\titlepage
\end{frame}

%%%%%%%%%%%%%%%%%%%%%%%%%%%%%%%%%%%%%%%%%%%%%%%%%%%%%%%%%%%%%%%%%%%%%%%%%%%%%%%%

\begin{frame}{Statics and Math }
	% centered minipage with text raggedright
	%cmini[width]{content}
	\cmini[0.8]{
		\begin{tcolorbox}[colframe=structure, colback=structure!15, left=0pt]
			% \raggedright
			\begin{itemize}
				\item Statics is all math! All but the most trivial statics problems require algebra and/or trigonometry and/or geometry to solve.
				\item[]\item The good news is that the math is not very complex. You don't need anything more advanced than high-school math to succeed in this course.
				\item[]\item We will do a quick review that should cover all the math you'll need for STCS 242.
			\end{itemize}
		\end{tcolorbox}

	}
\end{frame}

%%%%%%%%%%%%%%%%%%%%%%%%%%%%%%%%%%%%%%%%%%%%%%%%%%%%%%%%%%%%%%%%%%%%%%%%%%%%%%%%

\begin{frame}{Trigonometry}
	\centering
	Triangles are a very stable shape, and used often in engineering.
	\parm
	Triangles help avoid issues like this:
	\parm
	\tcbox[colback=white,boxsep=0pt,top=5pt,bottom=5pt,left=5pt,
	right=5pt, colframe=structure]{\fig[0.0625]{../../Figs/01MathReview/090517_042}}
	\parm
	Triangles mean we need trigonometry.
\end{frame}

%%%%%%%%%%%%%%%%%%%%%%%%%%%%%%%%%%%%%%%%%%%%%%%%%%%%%%%%%%%%%%%%%%%%%%%%%%%%%%%%

\begin{frame}{Right Triangle}
	% centered minipage with text raggedright
	%cmini[width]{content}
	\cmini[0.8]{
		\centering
		A right triangle is a triangle having one $90^\circ$ angle.
		\parb
		\only<1>{
			\tcbox[colback=white,boxsep=0pt,top=5pt,bottom=5pt,left=5pt,
			right=5pt, colframe=structure]{\fig[0.35]{../../Figs/01MathReview/rightTriangle}}
		}
		\only<2>{
			\tcbox[colback=white,boxsep=0pt,top=5pt,bottom=5pt,left=5pt,
			right=5pt, colframe=structure]{\fig[0.35]{../../Figs/01MathReview/rightTriangleA}}
		}
		\pause
		\parb
		Label the three sides $a$, $b$ and $c$. If we know the lengths of any two sides, we can calculate the length of the third side using the {\bfseries Pythagorean Theorem}:
		\parb
		% centered minipage with text raggedright
		%cmini[width]{content}
		\cmini[0.5]{
			\begin{tcolorbox}[colback=white,colframe=structure, title=Pythagorean Theorem]
				\[ a^2 = b^2 + c^2 \]
			\end{tcolorbox}
		}
	}
\end{frame}

%%%%%%%%%%%%%%%%%%%%%%%%%%%%%%%%%%%%%%%%%%%%%%%%%%%%%%%%%%%%%%%%%%%%%%%%%%%%%%%%

\begin{frame}{Calculations for Exercises}
	% centered minipage with text raggedright
	%cmini[width]{content}
	\cmini[0.8]{
		\begin{itemize}
			\item In general, it is difficult to measure objects more accurately than to three significant digits (or four, if the first digit is a 1) so given values to exercises are generally given to three (or four) significant digits.
			\item We cannot expect to get more accuracy from our result than from our given values at the beginning of the question so {\bfseries solutions should be correct to three significant digits (or four, if the first non-zero digit is~a~1)}
			\item Intermediate calculations will accumulate rounding errors quickly if we use only three significant digits and these can affect the final result. {\bfseries For intermediate calculations, use 5 or more significant digits.}\parm
			      When I write solutions on the board, I use 5 significant digits for intermediate calculations. You may use more if it is more convenient for you, i.e. if you are storing intermediate results in your calculator.
		\end{itemize}
	}

\end{frame}


%%%%%%%%%%%%%%%%%%%%%%%%%%%%%%%%%%%%%%%%%%%%%%%%%%%%%%%%%%%%%%%%%%%%%%%%%%%%%%%%

\begin{frame}{Right Triangle Exercises}
	% centered minipage with text raggedright
	%cmini[width]{content}
	\cmini[0.8]{
		\centering
		\tcbox[colback=white,boxsep=0pt,top=5pt,bottom=5pt,left=5pt,
		right=5pt, colframe=structure]{\fig[0.25]{../../Figs/01MathReview/MoS1a}}
		\parb
		\begin{enumerate}
			\item  Determine the lengths of $CE$ and $CB$
			      \setcounter{myexercisecounter}{\theenumi}
		\end{enumerate}


	}

	\vfill
	% answers, gray, centered, upside down
	\begin{center}
		\footnotesize
		\textcolor{gray}{
			\rotatebox[origin=c]{180}{
				($CE=4.80\text{ m},\, CD=3.25\text{ m} $ )
			}
		}
	\end{center}
\end{frame}

%%%%%%%%%%%%%%%%%%%%%%%%%%%%%%%%%%%%%%%%%%%%%%%%%%%%%%%%%%%%%%%%%%%%%%%%%%%%%%%%

\begin{frame}{More About Right Triangles}
	% centered minipage with text raggedright
	%cmini[width]{content}
	\cmini[0.8]{
		The sine, cosine and tangent trigonometrical functions relate an acute angle ($\theta$, in this example) in a right triangle to two of the sides of the triangle.
		\parm
		\begin{center}
			\tcbox[colback=white,boxsep=0pt,top=5pt,bottom=5pt,left=5pt,
			right=5pt, colframe=structure]{\fig[0.3]{../../Figs/01MathReview/rightTriangleB}}
		\end{center}
		The formul\ae{} are:
	}
	\begin{center}
		\begin{tcolorbox}[colback=white,boxsep=0pt,top=5pt,bottom=5pt,left=5pt,
			right=5pt, colframe=structure]
			$$
			\sin\theta = \frac{{\bm o}\text{\textcolor{gray}{pposite}}}{{\bm h}\text{\textcolor{gray}{ypotenuse}}}, \quad
			\cos\theta = \frac{{\bm a}\text{\textcolor{gray}{djacent}}}{{\bm h}\text{\textcolor{gray}{ypotenuse}}},
			\quad
			\tan\theta = \frac{{\bm o}\text{\textcolor{gray}{pposite}}}{{\bm a}\text{\textcolor{gray}{djacent}}}
			$$
		\end{tcolorbox}
		Remember: {\bfseries SOHCAHTOA}
	\end{center}
\end{frame}

%%%%%%%%%%%%%%%%%%%%%%%%%%%%%%%%%%%%%%%%%%%%%%%%%%%%%%%%%%%%%%%%%%%%%%%%%%%%%%%%

\begin{frame}{Right Triangle Exercises (2)}
	% centered minipage with text raggedright
	%cmini[width]{content}
	\cmini[0.85]{
		\centering
		\tcbox[colback=white,boxsep=0pt,top=5pt,bottom=5pt,left=5pt,
		right=5pt, colframe=structure]{\fig[0.2]{../../Figs/01MathReview/MoS1a}}
		\par
		\begin{enumerate}
			\setcounter{enumi}{\themyexercisecounter}
			\item Use the {\bfseries tangent} function to calculate $\angle CEF$.
			      {\footnotesize
			      	\textcolor{gray}{
			      		\rotatebox[origin=c, y=2.5pt]{180}{
			      			( $51.3^\circ$)
			      		}
			      	}
			      }			      

			\item From $\angle CEF$ just found ({\bfseries use the intermediate, 5 or more significant digit, form!}) and the {\bfseries sine} rule to verify the length of $CE$ found earlier.


			\item Use the {\bfseries cosine} function and the length of $CB$ found earlier to calculate the angle between $BC$ and the horizontal. {\footnotesize\textcolor{gray}{\rotatebox[origin=c, y=2.5pt]{180}{($22.6^\circ$ )}}}
			\item Use the {\bfseries tangent} function to verify the previous result.
			      \setcounter{myexercisecounter}{\theenumi}
		\end{enumerate}
	}
\end{frame}

%%%%%%%%%%%%%%%%%%%%%%%%%%%%%%%%%%%%%%%%%%%%%%%%%%%%%%%%%%%%%%%%%%%%%%%%%%%%%%%%

\begin{frame}{Triangles - Sine Rule}
	% centered minipage with text raggedright
	%cmini[width]{content}
	\cmini[0.8]{
		Not all triangles contain a right angle. To solve for these triangles (meaning finding the lengths of the side, and the angles in the triangle), we have to employ some different tools: 	 the {\bfseries sine rule} and (later) the {\bfseries cosine rule}
		\parm
		\centering
		\tcbox[colback=white,boxsep=0pt,top=5pt,bottom=5pt,left=5pt,
		right=5pt, colframe=structure]{\fig[0.25]{../../Figs/01MathReview/sineCosineRule}}
	}
	% centered minipage with text raggedright
	%cmini[width]{content}
	\cmini[0.9]{
		\mini[0.45]{
			\centering
			\begin{tcolorbox}[colback=white,colframe=structure, title=Sine Rule]
				\[ \frac{\sin A}{a} = \frac{\sin B}{b}=\frac{\sin C}{c} \]
			\end{tcolorbox}
		}
		or
		\mini[0.45]{
			\begin{tcolorbox}[colback=white,colframe=structure, title=Sine Rule]
				\[ \frac{a}{\sin A} = \frac{b}{\sin B}=\frac{c}{\sin C} \]
			\end{tcolorbox}
		}
	}
\end{frame}

%%%%%%%%%%%%%%%%%%%%%%%%%%%%%%%%%%%%%%%%%%%%%%%%%%%%%%%%%%%%%%%%%%%%%%%%%%%%%%%%

\begin{frame}{Triangles - Sine Rule Exercises}

	\mini[0.6]{
		\begin{enumerate}
			\setcounter{enumi}{\themyexercisecounter}
			\item Determine $\angle ABC$. {\footnotesize\textcolor{gray}{\rotatebox[origin=c, y=2.5pt]{180}{($95.8^\circ$)}}}\lb
			      Note that $82.2^\circ$ is also mathematically correct. From the diagram, though, it is `clear' that the angle is not less than $90^\circ$
			\item Determine $\angle ACB$. {\footnotesize\textcolor{gray}{\rotatebox[origin=c, y=2.5pt]{180}{($28.2^\circ$)}}}
			\item Determine the length of $AB$. {\footnotesize\textcolor{gray}{\rotatebox[origin=c, y=2.5pt]{180}{($2.85\text{ cm}$)}}}
			      \setcounter{myexercisecounter}{\theenumi}
		\end{enumerate}
	}
	\mini[0.35]{
		\centering
		\tcbox[colback=white,boxsep=0pt,top=5pt,bottom=5pt,left=5pt,
		right=5pt, colframe=structure]{\fig[0.25]{../../Figs/01MathReview/sineRule1}}
	}
\end{frame}

%%%%%%%%%%%%%%%%%%%%%%%%%%%%%%%%%%%%%%%%%%%%%%%%%%%%%%%%%%%%%%%%%%%%%%%%%%%%%%%%

\begin{frame}{Triangles - Cosine Rule}
	% centered minipage with text raggedright
	%cmini[width]{content}
	\cmini[0.8]{
		\centering
		\tcbox[colback=white,boxsep=0pt,top=5pt,bottom=5pt,left=5pt,
		right=5pt, colframe=structure]{\fig[0.2]{../../Figs/01MathReview/sineCosineRule}}
	}
	% centered minipage with text raggedright
	%cmini[width]{content}
	\cmini[0.6]{
		\begin{tcolorbox}[colback=white,colframe=structure, top=0pt,bottom=0pt,left=0pt,
			right=0pt,title=Cosine Rule]
			\begin{align*}
				a^2 & = b^2 + c^2 -2bc\cos A \\
				b^2 & = a^2 + c^2 -2ac\cos B \\
				c^2 & = a^2 + b^2 -2ab\cos C \\
			\end{align*}
		\end{tcolorbox}
		\parm
		The cosine rule is useful when you have all the sides of a triangle and want to find the angles.

	}
\end{frame}

%%%%%%%%%%%%%%%%%%%%%%%%%%%%%%%%%%%%%%%%%%%%%%%%%%%%%%%%%%%%%%%%%%%%%%%%%%%%%%%%

\begin{frame}{Triangles - Cosine Rule Exercises}

	\mini[0.5]{
		\begin{enumerate}
			\setcounter{enumi}{\themyexercisecounter}
			\item Determine $\angle ABC$, using the value for $AB$ found earlier
			\item Compare the value for $\angle ABC$ with the value calculated earlier. \lb Is it the same? It should be!
			      \setcounter{myexercisecounter}{\theenumi}
		\end{enumerate}
	}
	\mini[0.4]{
		\centering
		\tcbox[colback=white,boxsep=0pt,top=5pt,bottom=5pt,left=5pt,
		right=5pt, colframe=structure]{\fig[0.25]{../../Figs/01MathReview/cosineRule1}}
	}
\end{frame}

%%%%%%%%%%%%%%%%%%%%%%%%%%%%%%%%%%%%%%%%%%%%%%%%%%%%%%%%%%%%%%%%%%%%%%%%%%%%%%%%

\begin{frame}{Trig Identities}
	% centered minipage with text raggedright
	%cmini[width]{content}
	\cmini[0.8]{
		\begin{center}
			A couple of trig identities that may come in useful:
			\parm
			% centered minipage with text raggedright
			%cmini[width]{content}
			\cmini[0.6]{
				\begin{tcolorbox}[colback=white,colframe=structure, top=0pt,bottom=0pt,left=0pt,
					right=0pt]
					\begin{align*}
						\sin\left(180^\circ - \theta\right) & = \sin\theta \\
						\cos\left( - \theta\right)          & = \cos\theta \\
					\end{align*}
				\end{tcolorbox}
			}
			\parm
			For example,
			\begin{gather*}
				\sin(140^\circ) = \sin(40^\circ) = 0.64279 \\
				\cos(42^\circ) = \cos(-42^\circ) = 0.74314
			\end{gather*}
			This means we have to be careful with \lb inverse trigonometric functions:
			\begin{gather*}
				\sin^{-1}(0.64279) = 40^\circ \text{ \bfseries or } 140^\circ \\
				\cos^{-1}(0.74314) = 42^\circ \text{ \bfseries or } -42^\circ \\
			\end{gather*}
		\end{center}
	}
\end{frame}

%%%%%%%%%%%%%%%%%%%%%%%%%%%%%%%%%%%%%%%%%%%%%%%%%%%%%%%%%%%%%%%%%%%%%%%%%%%%%%%%

\begin{frame}{Similar Triangles}
	\begin{center}
		\tcbox[colback=white,boxsep=0pt,top=5pt,bottom=5pt,left=5pt,
		right=5pt, colframe=structure]{\fig[0.35]{../../Figs/01MathReview/similarTriangles}}
	\end{center}
	% centered minipage with text raggedright
	%cmini[width]{content}
	\cmini[0.8]{
		If triangles $ABC$ and $XYZ$ have the same angles, they are said to be {\bfseries similar}.
		\parm
		The ratios of the lengths of corresponding sides of similar triangles are equal:
		\parm
		\centering
		\tcbox[colback=white,boxsep=0pt,top=5pt,bottom=5pt,left=5pt,
		right=5pt, colframe=structure]{
			$ \frac{AB}{XY} = \frac{BC}{XZ} = \frac{AC}{YZ} $
		}
	}
\end{frame}

%%%%%%%%%%%%%%%%%%%%%%%%%%%%%%%%%%%%%%%%%%%%%%%%%%%%%%%%%%%%%%%%%%%%%%%%%%%%%%%%%%

\begin{frame}{Similar Triangles - Exercises}
	% left-aligned minipage with text raggedright
	%mini[width]{content}
	\mini[0.4]{
		$ABCD$ is a rigid (i.e., it does not deform) plate, pinned at $C$. \parm
		When horizontal force $P$ is applied at $A$, $ABCD$ rotates about $C$ and $A$ deflects 2.45~mm horizontally rightwards. \parm
		Assume that $BF$ remains horizontal and that $DE$ remains vertical.
		\only<1>{
			\begin{enumerate}
				\setcounter{enumi}{\themyexercisecounter}
				\item Determine $\delta_{BF}$, the change in length of $BF$.
				\item Determine $\delta_{DE}$, the change in length of $DE$.
				      %\setcounter{myexercisecounter}{\theenumi}
			\end{enumerate}
		}
		\only<2>{
			\begin{enumerate}
				\setcounter{enumi}{\themyexercisecounter}
				\item Determine $\delta_{BF}$, the change in length of $BF$.
				\item Determine $\delta_{DE}$, the change in length of $DE$.
				      \setcounter{myexercisecounter}{\theenumi}
			\end{enumerate}
		}
		\uncover<2>{
			% answers, gray, centered, upside down
			\begin{center}
				\footnotesize
				\textcolor{gray}{
					\rotatebox[origin=c]{180}{
						($\delta_{BF}=0.969\text{ mm},\,\delta_{DE}=1.319\text{ mm}$)
					}
				}
			\end{center}
		}
	}
	\hfill
	% left-aligned minipage with text raggedright
	%mini[width]{content}
	\mini[0.5]{
		\only<1>{
			\tcbox[colback=white,boxsep=0pt,top=5pt,bottom=5pt,left=5pt,
			right=5pt, colframe=structure]{\fig[0.25]{../../Figs/01MathReview/simTriangles}}
		}
		\only<2>{
			\tcbox[colback=white,boxsep=0pt,top=5pt,bottom=5pt,left=5pt,
			right=5pt, colframe=structure]{\fig[0.25]{../../Figs/01MathReview/simTrianglesB}}
		}

	}
\end{frame}

%%%%%%%%%%%%%%%%%%%%%%%%%%%%%%%%%%%%%%%%%%%%%%%%%%%%%%%%%%%%%%%%%%%%%%%%%%%%%%%%

\begin{frame}{Right Triangles and Trigonometric Functions - Exercises}
	\hspace{-0.5cm}
	% left-aligned minipage with text raggedright
	%mini[width]{content}
	\mini[0.6]{
		\begin{enumerate}
			\setcounter{enumi}{\themyexercisecounter}
			\item  Show that right triangles $\triangle ABC$, $\triangle ABD$ and $\triangle ACD$ all have the same angles (i.e. they are all similar).
			\item Given that $AC=100\text{ mm}$ and $AD=65\text{ mm}$, determine $\angle ACD$ and $\angle ABD$. {\footnotesize\textcolor{gray}{\rotatebox[origin=c, y=2.5pt]{180}{($\angle ACD=40.5^\circ, \, \angle ABD = 49.5^\circ$)}}}
			\item Find the remaining lengths: $AB$, $BD$ and $CD$. {\footnotesize\textcolor{gray}{\rotatebox[origin=c, y=2.5pt]{180}{($CD=76.0\text{ mm, }AB=85.5\text{ mm, }BD=55.6\text{ mm}$)}}}
			\item Verify your lengths found above by using the Pythagorean Theorem on $\triangle ABC$
			      \setcounter{myexercisecounter}{\theenumi}
		\end{enumerate}


	}
	\hfill
	% left-aligned minipage with text raggedright
	%mini[width]{content}
	\mini[0.39]{

		\tcbox[colback=white,boxsep=0pt,top=5pt,bottom=5pt,left=5pt,
		right=5pt, colframe=structure]{\fig[0.35]{../../Figs/01MathReview/simTrianglesC}}
	}
\end{frame}

%%%%%%%%%%%%%%%%%%%%%%%%%%%%%%%%%%%%%%%%%%%%%%%%%%%%%%%%%%%%%%%%%%%%%%%%%%%%%%%

\begin{frame}{Triangles and Trig Functions Exercise}
	\hspace{-.75cm}
	% left-aligned minipage with text raggedright
	%mini[width]{content}
	\mini[0.5]{
		This is a standard statics problem to determine the forces in ropes $AC$ and $BC$. But first we have to find the angles $\theta_{AC}$ and $\theta_{BC}$. This involves the use of the Pythagorean Theorem, the Sine and Cosine Rules, and one of the trigonometric functions.

		\begin{enumerate}
			\setcounter{enumi}{\themyexercisecounter}
			\item Find $\theta_{AC}$.
			\item Find $\theta_{BC}$.

		\end{enumerate}
		{\bfseries Don't give up yet!} Spend a few minutes and figure out how you can use the dimensions given to work towards finding angles $\theta_{AC}$ and $\theta_{BC}$.\parb
		\textcolor{gray}{If you are really stuck, there are hints on the next slide :)}
	}
	\hfill
	% left-aligned minipage with text raggedright
	%mini[width]{content}
	\mini[0.49]{
		\tcbox[colback=white,boxsep=0pt,top=5pt,bottom=5pt,left=5pt,
		right=5pt, colframe=structure]{\fig[0.575]{../../Figs/01MathReview/conc1}}
	}
\end{frame}

%%%%%%%%%%%%%%%%%%%%%%%%%%%%%%%%%%%%%%%%%%%%%%%%%%%%%%%%%%%%%%%%%%%%%%%%%%%%%%%

\begin{frame}{Triangles and Trig Functions Exercise Hints}
	\hspace{-.75cm}
	% left-aligned minipage with text raggedright
	%mini[width]{content}
	\mini[0.5]{
		\begin{itemize}
			\item Use the Pythagorean Theorem to find the distance between points $A$ and $B$. {\footnotesize\textcolor{gray}{\rotatebox[origin=c, y=2.5pt]{180}{(4.61\text{ m})}}}
			\item Use the Cosine Rule on $\triangle ABC$ to calculate $\angle ACB$. {\footnotesize\textcolor{gray}{\rotatebox[origin=c, y=2.5pt]{180}{($\angle ACB = 120.3^\circ$)}}}
			\item Use the Sine Rule on $\triangle ABC$ to calculate $\angle ABC$. {\footnotesize\textcolor{gray}{\rotatebox[origin=c, y=2.5pt]{180}{($\angle ABC=25.5^\circ$)}}}
			\item Use the tangent function to calculate $\angle OBA$. {\footnotesize\textcolor{gray}{\rotatebox[origin=c, y=2.5pt]{180}{($\angle OBA=12.26^\circ$)}}}
			\item $\angle OBC = \angle OBA + \angle ABC = \theta_{BC}$ (Alternate angles, also known as the Z-rule). {\footnotesize\textcolor{gray}{\rotatebox[origin=c, y=2.5pt]{180}{($\theta_{BC}=38.0^\circ$)}}}
			\item $\theta_{AC}=180^\circ-\left( \angle ACB + \theta_{BC} \right)$. {\footnotesize\textcolor{gray}{\rotatebox[origin=c, y=2.5pt]{180}{($\theta_{AC}=21.7^\circ$)}}}
		\end{itemize}

	}
	\hfill
	% left-aligned minipage with text raggedright
	%mini[width]{content}
	\mini[0.49]{
		\tcbox[colback=white,boxsep=0pt,top=5pt,bottom=5pt,left=5pt,
		right=5pt, colframe=structure]{\fig[0.575]{../../Figs/01MathReview/conc1}}
	}
\end{frame}

%%%%%%%%%%%%%%%%%%%%%%%%%%%%%%%%%%%%%%%%%%%%%%%%%%%%%%%%%%%%%%%%%%%%%%%%%%%%%%%

\begin{frame}{Algebraic Manipulation}

	% centered minipage with text raggedright
	%cmini[width]{content}
	\cmini[0.8]{
		We frequently need to solve an equation for a particular variable \lb(i.e. rearrange an equation to isolate a given variable)\parm
		For example:\parm
		\begin{tcolorbox}[colback=white,colframe=structure, top=0pt,bottom=0pt,left=5pt,
			right=5pt]
			\begin{align*}
				\text{Solve } \delta                           & = \dfrac{F\cdot L}{A\cdot E} \text{ for } E                           \\
				\intertext{Multiply both sides of the equation by $E/\delta$. Then:}
				\cancel{\delta}\cdot\frac {E}{\cancel{\delta}} & = \dfrac{F\cdot L}{A\cdot \cancel{E}} \cdot\frac {\cancel{E}}{\delta} \\\\
				E                                              & = \dfrac{F\cdot L}{A\cdot\delta}                                      \\
			\end{align*}
		\end{tcolorbox}
	}
\end{frame}

%%%%%%%%%%%%%%%%%%%%%%%%%%%%%%%%%%%%%%%%%%%%%%%%%%%%%%%%%%%%%%%%%%%%%%%%%%%%%%%

\begin{frame}{Algebraic Manipulation - Exercises }

	% centered minipage with text raggedright
	%cmini[width]{content}
	\cmini[0.85]{

		\begin{tcolorbox}[colback=white,colframe=structure, top=5pt,bottom=5pt,left=5pt,
			right=5pt]
			\begin{enumerate}
				\setcounter{enumi}{\themyexercisecounter}
				\item Solve \textcolor{Maroon}{$a^2=b^2+c^2$} for \textcolor{Maroon}{$b$}.\parm
				\item Solve \textcolor{Maroon}{$V=\frac43 \pi r^3$} for \textcolor{Maroon}{$r$}. \parm
				\item Solve \textcolor{Maroon}{$c^2=a^2+b^2-2bc\cos C$} for \textcolor{Maroon}{$\cos C$}.\parm
				\item Solve \textcolor{Maroon}{$b^2=a^2+c^2-2ac\cos B$} for \textcolor{Maroon}{$B$}.\parm
				\item This one is tricky!
				      One representation of the Hazen-Williams Equation for flow of water in a pipe is:
				      \textcolor{Maroon}{$$  Q=\frac{CD^{2.63}\left(\frac{h_L}{L}\right)^{0.54}}{279000}$$}\lb
				      Solve the equation for \textcolor{Maroon}{$h_L$}, then evaluate \textcolor{Maroon}{$h_L$} using the values \textcolor{Maroon}{$Q=135$, $C=120$, $D=202.7$ and $L=1200$}. \lb
				      \textcolor{gray}{(Answer: $h_L=105.5$, or check your result \href{http://www.eduk8r.org/hazenWilliamsFlowCalculator/index.html}{here})}
				      \setcounter{myexercisecounter}{\theenumi}
			\end{enumerate}
		\end{tcolorbox}
	}
\end{frame}

%%%%%%%%%%%%%%%%%%%%%%%%%%%%%%%%%%%%%%%%%%%%%%%%%%%%%%%%%%%%%%%%%%%%%%%%%%%%%%%

%%%%%%%%%%%%%%%%%%%%%%%%%%%%%%%%%%%%%%%%%%%%%%%%%%%%%%%%%%%%%%%%%%%%%%%%%%%%%%%

\begin{frame}{Simultaneous Equations}

	% left-aligned minipage with text raggedright
	%mini[width]{content}
	\mini[0.475]{
		\tcbox[colback=white,boxsep=0pt,top=5pt,bottom=5pt,left=5pt,
		right=5pt, colframe=structure]{\fig[0.5]{../../Figs/01MathReview/lights}}

	}
	\hfill
	% left-aligned minipage with text raggedright
	%mini[width]{content}
	\mini[0.475]{
		\tcbox[colback=white,boxsep=0pt,top=5pt,bottom=5pt,left=5pt,
		right=5pt, colframe=structure]{\fig[0.5]{../../Figs/01MathReview/conc1}}
	}
	% centered minipage with text raggedright
	%cmini[width]{content}
	\cmini[0.8]{
		Calculating the forces in $AC$ and $BC$ in each of the examples shown involves solving two equations in two unknowns (also known as solving a system of simultaneous equations).\parm
		We'll review how to do this.
	}

\end{frame}

%%%%%%%%%%%%%%%%%%%%%%%%%%%%%%%%%%%%%%%%%%%%%%%%%%%%%%%%%%%%%%%%%%%%%%%%%%%%%%%%



\begin{frame}{Solving Simultaneous Equations}
	% centered minipage with text raggedright
	%cmini[width]{content}
	\cmini[0.8]{
		This is a simple system of simultaneous equations:
		\begin{align}
			2x + 3y & = 7 \\
			6x-y    & = 1
		\end{align}
		Our objective is to find the value of $x$ and $y$ that satisfies both equations.\pause
		\parb
		Equation (1) is a straight line, with slope $-2/3$, that intersects the $x$ axis at $x=3.5$ and intersects the $y$ axis at $y=7/3$. \parm
		Equation (2) is a straight line, with slope $6$, that intersects the $x$ axis at $x=1/6$ and intersects the $y$ axis at  $y=-1$. \parm
		\pause
		These lines have different slopes, so they must intersect somewhere. At the point $(x, \, y)$ where they intersect, both equations are satisfied. This is the solution we're looking for.
	}
\end{frame}
%%%%%%%%%%%%%%%%%%%%%%%%%%%%%%%%%%%%%%%%%%%%%%%%%%%%%%%%%%%%%%%%%%%%%%%%%%%%%%%%

\begin{frame}{Solving Simultaneous Equations (cont'd)}

	% left-aligned minipage with text raggedright
	%mini[width]{content}
	\mini[0.55]{
		Graphically, the lines look like this. \parm
		It looks like the point at which they intersect is in the region of $(0.5,\, 2))$ \parm
		We can check whether this is the correct solution be substituting the values of $x=0.5$ and $y=2$ to see whether they satisfy both equations. (Do they?)\parm
		Generally,  for consistent access to accurate answers, we solve algebraically using a procedure called the Method of Subsitution.
	}
	\hfill
	% centered minipage with text raggedright
	%cmini[width]{content}
	\mini[0.4]{

		\begin{tcolorbox}[colback=white,colframe=structure, top=5pt,bottom=5pt,left=5pt,
			right=5pt]
			\begin{tikzpicture}
				\draw[very thin,color=gray] (-0.3,-2.3) grid (2.9,3.3);
				\draw[->] (-0.2,0) -- (3.2,0) node[right] {$x$};
				\draw[->] (0,-2.5) -- (0,3.8) node[above] {$y$};
				\draw[domain=-0.2:2.5, color=red, thick] plot[id=1] function{(7-2*x)/3}
				node[below] {$2x+3y=7$};
				\draw[domain=-0.2:0.8, color=blue, thick] plot[id=2] function{6*x-1}
				node[right] {$6x-y=1$};

			\end{tikzpicture}
		\end{tcolorbox}
	}

\end{frame}

%%%%%%%%%%%%%%%%%%%%%%%%%%%%%%%%%%%%%%%%%%%%%%%%%%%%%%%%%%%%%%%%%%%%%%%%%%%%%%%%



\begin{frame}{Solving Simultaneous Equations using the Method of Subsitution}
	\small
	% centered minipage with text raggedright
	%cmini[width]{content}
	\cmini[0.8]{
		\setcounter{equation}{0}
		\begin{align}
			2x + 3y & = 7 \\
			6x-y    & = 1
		\end{align}
		The process:
		\begin{enumerate}[(a)]
			\item  Choose an equation and solve for one of the variables.
			      Here I choose equation (2) and solve for the variable $y$.
			      \begin{align}
			      	y & = 6x-1
			      \end{align}
			\item Use equation (3) to substitute $6x-1$ wherever $y$ occurs in the other equation:
			      \begin{align*}
			      	2x+3(6x-1) & = 7   \\
			      	2x+18x-3   & = 7   \\
			      	20x        & = 10  \\
			      	x          & = 1/2
			      \end{align*}
			\item Substitute this value for $x$ in either of equation (1) or (2):
			      \begin{align*}
			      	6x-y & = 1 \\
			      	3d-y & = 1 \\
			      	y    & = 2
			      \end{align*}
			\item We have our solution: $(1/2, \, 2)$.

		\end{enumerate}
	}
\end{frame}

%%%%%%%%%%%%%%%%%%%%%%%%%%%%%%%%%%%%%%%%%%%%%%%%%%%%%%%%%%%%%%%%%%%%%%%%%%%%%%%%

\begin{frame}{Typical Statics System of Equations}
	% left-aligned minipage with text raggedright
	%mini[width]{content}
	\mini[0.475]{
		In an earlier exercise, we calculated that $\theta_{AC}=21.661^\circ$ and
		$\theta_{BC}=38.049^\circ$. \parm
		(Note that we are using 5 significant digits here because we will be using these values for intermediate calculations.)

	}
	\hfill
	% left-aligned minipage with text raggedright
	%mini[width]{content}
	\mini[0.475]{
		\tcbox[colback=white,boxsep=0pt,top=5pt,bottom=5pt,left=5pt,
		right=5pt, colframe=structure]{\fig[0.5]{../../Figs/01MathReview/conc1}}
	}
	% centered minipage with text raggedright
	%cmini[width]{content}
	\cmini[0.8]{
		The system shown yields the following two equations in the two unknowns $F_{AC}$ and $F_{BC}$:
		\begin{align}
			\setcounter{equation}{0}
			F_{AC}\sin\left(21.661^\circ\right) + F_{BC}\sin\left(38.049^\circ\right) & = 2011.1 \\
			F_{BC}\cos\left(38.049^\circ\right)-F_{AC}\cos\left(21.661^\circ\right)   & = 0
		\end{align}
		{\bfseries Ouch!}

	}

\end{frame}

%%%%%%%%%%%%%%%%%%%%%%%%%%%%%%%%%%%%%%%%%%%%%%%%%%%%%%%%%%%%%%%%%%%%%%%%%%%%%%%%

\begin{frame}{Typical Statics System of Equations}
	% left-aligned minipage with text raggedright
	%mini[width]{content}
	\mini[0.475]{
		In an earlier exercise, we calculated that $\theta_{AC}=21.661^\circ$ and
		$\theta_{BC}=38.049^\circ$. \parm
		(Note that we are using 5 significant digits here because we will be using these values for intermediate calculations.)

	}
	\hfill
	% left-aligned minipage with text raggedright
	%mini[width]{content}
	\mini[0.475]{
		\tcbox[colback=white,boxsep=0pt,top=5pt,bottom=5pt,left=5pt,
		right=5pt, colframe=structure]{\fig[0.5]{../../Figs/01MathReview/conc1}}
	}
	% centered minipage with text raggedright
	%cmini[width]{content}
	\cmini[0.8]{
		The system shown yields the following two equations in the two unknowns $F_{AC}$ and $F_{BC}$:
		\begin{align}
			\setcounter{equation}{0}
			F_{AC}\sin\left(21.661^\circ\right) + F_{BC}\sin\left(38.049^\circ\right) & = 2011.1 \\
			F_{BC}\cos\left(38.049^\circ\right)-F_{AC}\cos\left(21.661^\circ\right)   & = 0
		\end{align}
		{\bfseries Ouch!} This looks much harder than the previous example :(

	}

\end{frame}

%%%%%%%%%%%%%%%%%%%%%%%%%%%%%%%%%%%%%%%%%%%%%%%%%%%%%%%%%%%%%%%%%%%%%%%%%%%%%%%%

\begin{frame}{Typical Statics System of Equations Exercise}


	% centered minipage with text raggedright
	%cmini[width]{content}
	\cmini[0.8]{
		\begin{align}
			\setcounter{equation}{0}
			F_{AC}\sin\left(21.661^\circ\right) + F_{BC}\sin\left(38.049^\circ\right) & = 2011.1 \\
			F_{BC}\cos\left(38.049^\circ\right)-F_{AC}\cos\left(21.661^\circ\right)   & = 0
		\end{align}
		Fortunately, it only looks harder. $F_{AC}$ and $F_{BC}$ are variables, just like $x$ and $y$ in the simple example.
		The sines and cosines are just numbers.

		\begin{align}
			0.36911x + 0.61633y & = 2011.1 \\
			0.78748y - 0.92938x & = 0
		\end{align}

		This is the same system, with $x$ replacing $F_{AC}$, $y$ replacing $F_{BC}$ and the trigonemetric functions evaluated.

		\begin{enumerate}
			\setcounter{enumi}{\themyexercisecounter}
			\item Solve for $x$ (which is $F_{AC}$ in the original system) {\footnotesize\textcolor{gray}{\rotatebox[origin=c, y=2.5pt]{180}{(1834)}}}\\
			\item Solve for $y$ (which is $F_{BC}$ in the original system) {\footnotesize\textcolor{gray}{\rotatebox[origin=c, y=2.5pt]{180}{(2160)}}}
			      \setcounter{myexercisecounter}{\theenumi}
		\end{enumerate}
	}
\end{frame}

%%%%%%%%%%%%%%%%%%%%%%%%%%%%%%%%%%%%%%%%%%%%%%%%%%%%%%%%%%%%%%%%%%%%%%%%%%%%%%%%

\begin{frame}{Typical Statics System of Equations (2)}

	% left-aligned minipage with text raggedright
	%mini[width]{content}
	\mini[0.35]{
		\tcbox[colback=white,boxsep=0pt,top=5pt,bottom=5pt,left=5pt,
		right=5pt, colframe=structure]{\fig[0.4]{../../Figs/01MathReview/lights}}
	}
	\hfill
	% centered minipage with text raggedright
	%cmini[width]{content}
	\mini[0.6]{
		The system shown yields the following two equations in the two unknowns $F_{AC}$ and $F_{BC}$:
		\begin{align}
			\setcounter{equation}{0}
			F_{BC}\sin15^\circ + F_{AC}\cos35^\circ + 1030.1 & = 0 \\
			F_{BC}\cos 15^\circ-F_{AC}\sin35^\circ           & = 0
		\end{align}
		\parb
		\begin{enumerate}
			\setcounter{enumi}{\themyexercisecounter}
			\item Determine $F_{AC}$ {\footnotesize\textcolor{gray}{\rotatebox[origin=c, y=2.5pt]{180}{(-1548)}}}\\
			\item Determine $F_{BC}$ {\footnotesize\textcolor{gray}{\rotatebox[origin=c, y=2.5pt]{180}{(919)}}}
			      \setcounter{myexercisecounter}{\theenumi}
		\end{enumerate}
	}
	% centered minipage with text raggedright
	%cmini[width]{content}
	\cmini[0.8]{
		\parb
		{\bfseries Note}: Simultaneous equations can also be solved using special functions on your calculator. This is allowed on quizzes and exams. Read your calculator manual!
	}


\end{frame}



\end{document}
