\documentclass[12pt,oneside]{article}

\usepackage{xcolor}
\usepackage{cancel}
\usepackage{bm}
\usepackage{graphicx}
\usepackage[x11names, svgnames]{xcolor} % for colors in handouts, auto loaded in Beamer?
\usepackage{tikz}
\usetikzlibrary{arrows.meta, math, calc, shadows}
\usetikzlibrary{decorations.markings, decorations.fractals, decorations.text} % for chain, etc.
\usetikzlibrary{intersections}
\usepackage{pgfmath}
\usepackage{ifthen}
\usepgfmodule{oo}
\usepgflibrary{shadings}
% \usetikzlibrary{decorations.shapes}
\usepackage[many]{tcolorbox}
\usepackage[absolute,overlay,showboxes]{textpos}
% \usepackage{textpos}
% \textblockorigin{0.0cm}{0.0cm}  %start all at upper left corner
\TPshowboxesfalse

\newcommand\lb{\linebreak}
\newcommand\Ra{\Rightarrow}
\newcommand\cd{\!\cdot\!}
\newcommand\x{\!\times\!}
\newcommand\pars{\par\smallskip}
\newcommand\parm{\par\medskip}
\newcommand\parb{\par\bigskip}
\renewcommand{\deg}{^\circ}

% counter for resuming enumerated list numbers
\newcounter{resumeenumi}
\newcommand{\suspend}{\setcounter{resumeenumi}{\theenumi}}
\newcommand{\resume}{\setcounter{enumi}{\theresumeenumi}}



% https://tex.stackexchange.com/questions/33703/extract-x-y-coordinate-of-an-arbitrary-point-in-tikz
\makeatletter
\providecommand{\gettikzxy}[3]{%
	\tikz@scan@one@point\pgfutil@firstofone#1\relax
	\edef#2{\the\pgf@x}%
	\edef#3{\the\pgf@y}%
}
\makeatother

\makeatletter
\newcommand{\verbatimfont}[1]{\def\verbatim@font{#1}}%
\makeatother

%%%%%%%%%%%%%%%%%%%%%%%%%%%%%%%%%%%%%%%%%%%%%%%%%%%%%%%%%%%%%%%%%%%%%%%%%%%%%%%%


\newcommand{\tb}[4][0.8]{
	\begin{textblock*}{#1}(#2, #3)
		% \raggedright
		#4
	\end{textblock*}
}

\newtcolorbox{statsbox}[2][] { 
  colback=white,
  colbacktitle=structure,
  colframe=structure,
  coltitle=white,  
  top=0.25cm,
	bottom=0.125cm,
	left=0mm,
	right=0mm,
  % fonttitle=\itshape\rmfamily,
  halign=flush left, 
  enhanced,
  drop fuzzy shadow,
  attach boxed title to top left={xshift=3.5mm, yshift=-2mm},
  title={#2}, #1}
\newtcolorbox{redbox}{colback=white, colframe=structure, enhanced, drop fuzzy shadow}
\newtcolorbox{titledbox}[1]{colback=white,colframe=structure,title={#1}}
\newtcbox{\tcb}[1][]{colback=white,boxsep=0pt,top=5pt,bottom=5pt,left=5pt,
		right=5pt, colframe=structure,  enhanced, drop fuzzy shadow, #1}
% tcb title
\newtcbox{\tcbt}[2][]{colback=white,boxsep=0pt,top=5pt,bottom=5pt,left=5pt,
		right=5pt, colframe=structure, enhanced, drop fuzzy shadow,  title={#2}, #1}
% tcb left title
\newtcbox{\tcbtl}[2][]{ colback=white,
  colbacktitle=structure,
  colframe=structure,
  coltitle=white,  
  top=0.25cm,
	bottom=0.125cm,
	left=0mm,
	right=0mm,
  % fonttitle=\bfseries,
  halign=flush left, 
  enhanced,
  drop fuzzy shadow,
  attach boxed title to top left={xshift=3.5mm, yshift=-2mm}, 
	title={#2}, #1}

\newtcbtheorem{myexam}{Example}%
{
	enhanced,
	colback=white,
	colframe=structure,
	% fonttitle=\bfseries,
	fonttitle=\itshape\rmfamily,
	drop fuzzy shadow,
	%description font=\mdseries\itshape,
	attach boxed title to top left={yshift=-2mm, xshift=5mm},
	colbacktitle=structure
	}{exam}% then \pageref{exer:theoexample} references the theo

% \newcommand{\myexample}[2][red]{
% 	% \tcb\tcbset{theostyle/.style={colframe=red,colbacktitle=yellow}}
% 	\begin{myexam}{}{}
% 		#2
% 	\end{myexam}
% 	% \tcbset{colframe=structure,colbacktitle=structure}
% }

\newtcbtheorem{myexer}{Exercise}%
{
	enhanced,
	colback=white,
	colframe=structure,
	% fonttitle=\bfseries,
	drop fuzzy shadow,
	fonttitle=\itshape\rmfamily,
	% description font=\mdseries\itshape,
	attach boxed title to top left={yshift=-2mm, xshift=5mm},
	colbacktitle=structure
	}{exer}



\newcommand{\mini}[2][0.8]{
	\begin{minipage}[c]{#1\columnwidth}
		\raggedright
		#2
	\end{minipage}
}
\newcommand{\minit}[2][0.8]{
	\begin{minipage}[t]{#1\columnwidth}
		% \raggedright
		#2
	\end{minipage}
}

% centered minipage with text \raggedright
%\cmini[width]{content}
\newcommand{\cmini}[2][0.8]{
	\begin{center}
		\begin{minipage}{#1\columnwidth}
			\raggedright
			#2
		\end{minipage}
	\end{center}
}



\newcommand{\fig}[2][1]{% scaled graphic
	\includegraphics[scale=#1]{#2}
}

% centred framed colored box black border
%\cbox[width]{content}
\newcommand{\cbox}[2][1]{% framed centered color box
	\setlength\fboxsep{5mm}
	\setlength\fboxrule{.2 mm}
	\begin{center}
		\fcolorbox{black}{white}{
			\vspace{-0.5cm}
			\begin{minipage}{#1\columnwidth}
				\raggedright
				#2
			\end{minipage}
		}
	\end{center}
	\setlength\fboxsep{0cm}
}

\newcommand{\cfig}[2][1]{% centred, scaled graphic
	\begin{center}
		\includegraphics[scale=#1]{#2}
	\end{center}
}






 \definecolor{saitPurple}{RGB}{112,40,119}
 \definecolor{statsMaroon}{rgb}{0.55, 0, 0}
 \definecolor{saitMaroon}{rgb}{0.55, 0, 0}
 \definecolor{statsRed}{RGB}{224,38,37}
 \definecolor{saitRed}{RGB}{224,38,37}
 \definecolor{saitBlue}{rgb}{0, 0.59, 0.85}
 \definecolor{statsBlue}{rgb}{0, 0.59, 0.85}
 \definecolor{statsDeepBlue}{RGB}{0, 99, 167}
 \definecolor{saitDeepBlue}{RGB}{0, 99, 167}
 \definecolor{saitDeepBlue}{RGB}{0, 99, 167}
 \definecolor{LightGrey}{RGB}{200,200,200}
%  \definecolor{boxBG}{RGB}{236, 227, 227}
%  \definecolor{boxBG}{RGB}{242, 233, 223}
%\Member{startpt}{endpt}{outer fill color}{inner fill color}{stroke}{height}{radius}{linewidth}
\providecommand{\Member}[8]{
  % name the points
  \coordinate(start) at (#1);
  \coordinate(end) at (#2);
  \edef\ofill{#3}%
  \edef\ifill{#4}%
  \edef\stroke{#5}%
  \edef\height{#6} % cm
  \edef\radius{#7} % cm
  \edef\linewidth{#8} % mm

  \coordinate(delta) at ($ (end)-(start) $);
  \gettikzxy{(delta)}{\dx}{\dy}
  \gettikzxy{(start)}{\sx}{\sy}
  \pgfmathparse{veclen(\dx, \dy)} \let\length\pgfmathresult

  \pgfmathparse{\dx==0}%
  % \ifnum low-level TeX for integers
  \ifnum\pgfmathresult=1 % \dx == 0
    \pgfmathsetmacro{\rot}{\dy > 0 ? 90 : -90}
  \else
    \pgfmathsetmacro{\rot}{\dx > 0 ? atan(\dy / \dx) : 180 + atan(\dy / \dx)}
  \fi

  
   
  \shadedraw[transform canvas = { rotate around = {\rot:(\sx,\sy)}}, line width = \linewidth, rounded corners = \radius mm, top color = \ofill, bottom color = \ofill, middle color = \ifill, draw = \stroke] ($ (start)+(-0.5*\height, 0.5*\height) $) -- ++(\height cm +\length pt, 0 ) -- ++(0, -\height) -- ++ (-\height cm -\length pt, 0) -- cycle;


  \shadedraw[ball color = \ofill!50!\ifill, draw = \stroke] (start) circle (\height/8);
  \shadedraw[ball color = \ofill!50!\ifill, draw = \stroke] (end) circle (\height/8);
  %  \pgfresetboundingbox

  
  


}


\newcommand{\PC}[6][0]{%
  \edef\lrotate{#1}%
  \edef\lpin{#2}%
  \edef\lfill{#3}%
  \edef\ldraw{#4}%
  \edef\lscale{#5}%
  \edef\lwidth{#6}% mm
  \edef\h{1}%
  \edef\r{0.3}%
  \begin{scope}[scale=\lscale, rotate=\lrotate]
	\filldraw[draw=\ldraw, fill=\lfill, line width=\lwidth mm] ($ (\lpin) + (0.201*\h+1.0353*\r ,-0.75*\h) $) -- ++(105: 0.77646*\h+0.26795*\r) arc (15:165:\r) -- ++(-105:0.77646*\h+0.26795*\r) -- cycle;

	\shadedraw[ball color=\lfill, draw=\ldraw, line width = \lwidth mm] (\lpin) circle (1.5mm);

	\filldraw[rounded corners=\lscale pt, draw=\ldraw, fill=\lfill, line width=\lwidth mm] ($ (\lpin) - (1,1) $) rectangle +(2,0.25);
  \end{scope}%
}




\renewcommand{\familydefault}{\sfdefault} 

\usepackage{mathpazo}
\usepackage[letterpaper]{geometry}
\geometry{verbose,tmargin=0.25in,bmargin=0.5in,lmargin=1in,rmargin=1.15in}
\hfuzz 50pt
\setlength{\parindent}{0pt}
\Large

\begin{document}

%%%%%%%%%%%%%%%%%%%%%%%%%%%%%%%%%%%%%%%%%%%%%%%%%%%%%%%%%%%%%%%%%%%%%%%%%%%%%%%%%%%%%%%%%%%%%%%%%%%%
% page 1
%%%%%%%%%%%%%%%%%%%%%%%%%%%%%%%%%%%%%%%%%%%%%%%%%%%%%%%%%%%%%%%%%%%%%%%%%%%%%%%%%%%%%%%%%%%%%%%%%%%%



% \begin{textblock*}{8in}(0in, 0.25in)
	
	% \cbox{
		
		% \mini[0.5]{
			% \cbox{
			\centering
			\tikz{%color
			\large
			\coordinate (O) at (0,0);
			
			\coordinate (A) at ($ (O)+(90:5) $); %45
			\coordinate (b) at ($ (O)+(60:5) $); %45
			\coordinate (C) at ($ (O)+(30:5) $); %45
			\coordinate (d) at ($ (O)+(0:5) $); %30
			\coordinate (E) at ($ (O)+(-30:5) $); %30
			\coordinate (f) at ($ (O)+(-90:5) $); %30
			\coordinate (G) at ($ (O)+(-150:5) $); %30
			\coordinate (h) at ($ (O)+(150:5) $); %45

			\path[name path=Ob] (O)--(b);
			\path[name path=AC] (A)--(C);
			\path[name path=Od] (O)--(d);
			\path[name path=CE] (C)--(E);
			\path[name path=Of] (O)--(f);
			\path[name path=EG] (E)--(G);
			\path[name path=Oh] (O)--(h);
			\path[name path=GA] (G)--(A);

			\path[name intersections = {of=Ob and AC, by=B}];
			\path[name intersections = {of=Od and CE, by=D}];
			\path[name intersections = {of=Of and EG, by=F}];
			\path[name intersections = {of=Oh and GA, by=H}];
			% \path[name intersections = {of=AC and BC, by=C}];
			% \path[name intersections = {of=AC and BC, by=C}];
			
			\filldraw [fill=OliveDrab2] (A)--(C)--(E)--(G)--cycle;
			\filldraw [fill=white] (H)--(D)--(F)--cycle;
			
			
			
			
			
			
			
			\draw[thin, LightGrey] (O) circle (5cm);

			% \fill (A) circle (0.5mm);
			% \fill (B) circle (0.5mm);
			% \fill (C) circle (0.5mm);
			% \fill (D) circle (0.5mm);
			% \fill (E) circle (0.5mm);
			% \fill (F) circle (0.5mm);
			% \fill (G) circle (0.5mm);
			% \fill (H) circle (0.5mm);

			\draw (O)--(A);
			\draw (O)--(B);
			\draw (O)--(C);
			\draw (O)--(D);
			\draw (O)--(E);
			\draw (O)--(F);
			\draw (O)--(G);
			\draw (O)--(H);		

			\node at ($ (O)!1.1!(A) $) {$A$};
			\node at ($ (O)!1.1!(B) $) {$B$};
			\node at ($ (O)!1.1!(C) $) {$C$};
			\node at ($ (O)!1.1!(D) $) {$D$};
			\node at ($ (O)!1.1!(E) $) {$E$};
			\node at ($ (O)!1.1!(F) $) {$F$};
			\node at ($ (O)!1.1!(G) $) {$G$};
			\node at ($ (O)!1.2!(H) $) {$H$};

			% \node at ($ (O)+(16:1.25) $) {$\frac{\pi}{6}$};
			% \node at ($ (O)+(46:1.25) $) {$\frac{\pi}{6}$};
			% \node at ($ (O)+(76:1.25) $) {$\frac{\pi}{6}$};
			% \node at ($ (O)+(-16:1.25) $) {$\frac{\pi}{6}$};
			% \node at ($ (O)+(120:1.25) $) {$\frac{\pi}{3}$};
			% \node at ($ (O)+(180:1.25) $) {$\frac{\pi}{3}$};
			% \node at ($ (O)+(240:1.25) $) {$\frac{\pi}{3}$};
			% \node at ($ (O)+(-60:1.25) $) {$\frac{\pi}{3}$};

			\draw ($(O)+(90:0.675)$) arc (90:60:0.675);
			\draw ($(O)+(60:0.75)$) arc (60:30:0.75);
			\draw ($(O)+(30:0.675)$) arc (30:0:0.675);
			\draw ($(O)+(0:0.75)$) arc (0:-30:0.75);

			\draw ($(O)+(90:0.675)$) arc (90:150:0.675);
			\draw ($(O)+(90:0.75)$) arc (90:150:0.75);
			\draw ($(O)+(150:0.825)$) arc (150:210:0.825);
			\draw ($(O)+(150:0.75)$) arc (150:210:0.75);
			\draw ($(O)+(210:0.675)$) arc (210:270:0.675);
			\draw ($(O)+(210:0.75)$) arc (210:270:0.75);
			\draw ($(O)+(270:0.825)$) arc (270:330:0.825);
			\draw ($(O)+(270:0.75)$) arc (270:330:0.75);

			\node[left, fill=white, inner sep=1mm,xshift=-0.5mm] at(O) {$O$};
			\fill (O) circle (1mm);		
		}
		\centering\parb
		The vertices of quadrilateral $ACEG$ lie on a circle with centre $O$.\parb
		$ \angle AOB = \angle BOC = \angle COD = \angle DOE $ and $ \angle EOF = \angle FOG = \angle GOH = \angle HOA $ \parb
		What fraction of $ACEG$'s area does it lose when $\triangle DFH$ is removed?
		
		
	%  }}

	 
	
\end{document}