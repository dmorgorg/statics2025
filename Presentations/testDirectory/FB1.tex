\documentclass[10pt,oneside]{article}

\usepackage{xcolor}
\usepackage{cancel}
\usepackage{bm}
\usepackage{graphicx}
\usepackage{hyperref}
\usepackage{adjustbox}
\hypersetup{colorlinks, allcolors=.,urlcolor=structure}
\usepackage{booktabs}  % for top and bottom spacing in table cells, \addlinespace
\usepackage[x11names, svgnames]{xcolor} % for colors in handouts, auto loaded in Beamer?
\usepackage{tikz}
\usetikzlibrary{arrows.meta, math, calc, shadows,bending}
\usetikzlibrary{decorations.markings, decorations.fractals, decorations.text} % for chain, etc.
\usetikzlibrary{intersections}
\usepackage{pgfmath}
\usepackage{ifthen}
\usepgfmodule{oo}
\usetikzlibrary{shadings}
% \usetikzlibrary{decorations.shapes}
\usepackage[many]{tcolorbox}
\tcbuselibrary{skins} % for image boxes
\usepackage[absolute,overlay,showboxes]{textpos}
% \usepackage{textpos}
% \textblockorigin{0.0cm}{0.0cm}  %start all at upper left corner
\TPshowboxesfalse

\newcommand\lb{\linebreak}
\newcommand\Ra{\Rightarrow}
\newcommand\cd{\!\cdot\!}
\newcommand\x{\!\times\!}
\newcommand\pars{\par\smallskip}
\newcommand\parm{\par\medskip}
\newcommand\parb{\par\bigskip}
\renewcommand{\deg}{^\circ}

% counter for resuming enumerated list numbers
\newcounter{resumeenumi}
\newcommand{\suspend}{\setcounter{resumeenumi}{\theenumi}}
\newcommand{\resume}{\setcounter{enumi}{\theresumeenumi}}



% https://tex.stackexchange.com/questions/33703/extract-x-y-coordinate-of-an-arbitrary-point-in-tikz
\makeatletter
\providecommand{\gettikzxy}[3]{%
	\tikz@scan@one@point\pgfutil@firstofone#1\relax
	\edef#2{\the\pgf@x}%
	\edef#3{\the\pgf@y}%
}
\makeatother

\makeatletter
\newcommand{\verbatimfont}[1]{\def\verbatim@font{#1}}%
\makeatother

%%%%%%%%%%%%%%%%%%%%%%%%%%%%%%%%%%%%%%%%%%%%%%%%%%%%%%%%%%%%%%%%%%%%%%%%%%%%%%%%


% \newcommand{\tb}[4][0.8]{
% 	\begin{textblock*}{#1}(#2, #3)
% 		\raggedright
% 		#4
% 	\end{textblock*}
% }

% \def\tb

\newtcolorbox{statsbox}[2][] { 
  colback=white,
  colbacktitle=structure,
  colframe=structure,
  coltitle=white,  
  top=0.25cm,
	bottom=0.125cm,
	left=0mm,
	right=0mm,
  % fonttitle=\itshape\rmfamily,
  halign=flush left, 
  enhanced,
  drop fuzzy shadow,
  attach boxed title to top left={xshift=3.5mm, yshift=-2mm},
  title={#2}, #1}
\newtcolorbox{redbox}{colback=white, colframe=structure, enhanced, drop fuzzy shadow}
\newtcolorbox{titledbox}[1]{colback=white,colframe=structure,title={#1}}
\newtcbox{\tcb}[1][]{colback=white,boxsep=0pt,top=0.5cm,bottom=0.5cm,left=0.5cm,
		right=0.5cm, colframe=structure,  enhanced, drop fuzzy shadow, #1}
\newtcbox{\tcbfig}[1][1]{colback=white,boxsep=0pt,top=0.5cm,bottom=0.5cm,left=0.5cm,
		right=0.5cm, colframe=structure,  enhanced, drop fuzzy shadow, #1}
% tcb title
\newtcbox{\tcbt}[2][]{colback=white,boxsep=0pt,top=5pt,bottom=5pt,left=5pt,
		right=5pt, colframe=structure, enhanced, drop fuzzy shadow,  title={#2}, #1}
% tcb left title
\newtcbox{\tcbtl}[2][]{ colback=white,
  colbacktitle=structure,
  colframe=structure,
  coltitle=white,  
  top=0.25cm,
	bottom=0.125cm,
	left=0mm,
	right=0mm,
  % fonttitle=\bfseries,
  halign=flush left, 
  enhanced,
  drop fuzzy shadow,
  attach boxed title to top left={xshift=3.5mm, yshift=-2mm}, 
	title={#2}, #1}

\newtcbtheorem{myexam}{Example}%
{
	enhanced,
	colback=white,
	colframe=structure,
	% fonttitle=\bfseries,
	fonttitle=\itshape\rmfamily,
	drop fuzzy shadow,
	%description font=\mdseries\itshape,
	attach boxed title to top left={yshift=-2mm, xshift=5mm},
	colbacktitle=structure
	}{exam}% then \pageref{exer:theoexample} references the theo

% \newcommand{\myexample}[2][red]{
% 	% \tcb\tcbset{theostyle/.style={colframe=red,colbacktitle=yellow}}
% 	\begin{myexam}{}{}
% 		#2
% 	\end{myexam}
% 	% \tcbset{colframe=structure,colbacktitle=structure}
% }

\newtcbtheorem{myexer}{Exercise}%
{
	enhanced,
	colback=white,
	colframe=structure,
	% fonttitle=\bfseries,
	drop fuzzy shadow,
	fonttitle=\itshape\rmfamily,
	% description font=\mdseries\itshape,
	attach boxed title to top left={yshift=-2mm, xshift=5mm},
	colbacktitle=structure
	}{exer}



\newcommand{\mini}[2][0.8]{
	\begin{minipage}[c]{#1\columnwidth}
		\raggedright
		#2
	\end{minipage}
}
\newcommand{\minit}[2][0.8]{
	\begin{minipage}[t]{#1\columnwidth}
		% \raggedright
		#2
	\end{minipage}
}

% centered minipage with text \raggedright
%\cmini[width]{content}
\newcommand{\cmini}[2][0.8]{
	\begin{center}
		\begin{minipage}{#1\columnwidth}
			\raggedright
			#2
		\end{minipage}
	\end{center}
}

\newcommand{\fig}[2][1]{% scaled graphic
	\includegraphics[scale=#1]{#2}
}

% centred framed box black border
%\cbox[width]{content}
\newcommand{\cbox}[2][1]{% framed centered color box
	\setlength\fboxsep{5mm}
	\setlength\fboxrule{.2 mm}
	\begin{center}
		\fcolorbox{black}{white}{
			\vspace{-0.5cm}
			\begin{minipage}{#1\columnwidth}
				\raggedright
				#2
			\end{minipage}
		}
	\end{center}
	\setlength\fboxsep{0cm}
}

\newcommand{\ccbox}[4][1]{% framed centered color box
	\setlength\fboxsep{5mm}
	\setlength\fboxrule{.2 mm}
	\begin{center}
		\fcolorbox{#2}{#3}{
			% \vspace{-0.5cm}
			\begin{minipage}{#1\columnwidth}
				\vspace{-0.25cm}
				\raggedright				
				#4
				\vspace{-0.325cm}
			\end{minipage}
		}
	\end{center}
	\setlength\fboxsep{0cm}
}

\newcommand{\cfig}[2][1]{% centred, scaled graphic
	\begin{center}
		% \fcolorbox{structure}{white}{
		\tcbincludegraphics{
			\includegraphics[scale=#1]{#2}
		}
	\end{center}
}

% figure with tight border for photos
% \cfigb[saitMaroon]{borderwidth with unit}{scale}{image}
\newcommand{\stcsfig}[2][1]{
	% \usepackage{adjustbox}
	% \setlength{\fboxrule}{1pt}
	\begin{center}
		\tcbincludegraphics[width=#1\textwidth, boxrule=2pt, top=-3pt, right=-3pt, left=-3pt, bottom=-3pt,colframe=structure, sharp corners, enhanced, drop fuzzy shadow]{#2}
	\end{center}
}






 \definecolor{saitPurple}{RGB}{112,40,119}
 \definecolor{statsMaroon}{rgb}{0.55, 0, 0}
 \definecolor{saitMaroon}{rgb}{0.55, 0, 0}
 \definecolor{saitRed}{RGB}{224,38,37}
 \definecolor{saitBlue}{rgb}{0, 0.59, 0.85}
 \definecolor{statsDeepBlue}{RGB}{0, 99, 167}
 \definecolor{saitDeepBlue}{RGB}{0, 99, 167}
 \definecolor{LightGrey}{RGB}{200,200,200}
%  \definecolor{boxBG}{RGB}{236, 227, 227}
%  \definecolor{boxBG}{RGB}{242, 233, 223}
%\Member{startpt}{endpt}{outer fill color}{inner fill color}{stroke}{height}{radius}{linewidth}
\providecommand{\Member}[8]{
  % name the points
  \coordinate(start) at (#1);
  \coordinate(end) at (#2);
  \edef\ofill{#3}%
  \edef\ifill{#4}%
  \edef\stroke{#5}%
  \edef\height{#6} % cm
  \edef\radius{#7} % cm
  \edef\linewidth{#8} % mm

  \coordinate(delta) at ($ (end)-(start) $);
  \gettikzxy{(delta)}{\dx}{\dy}
  \gettikzxy{(start)}{\sx}{\sy}
  \pgfmathparse{veclen(\dx, \dy)} \let\length\pgfmathresult

  \pgfmathparse{\dx==0}%
  % \ifnum low-level TeX for integers
  \ifnum\pgfmathresult=1 % \dx == 0
    \pgfmathsetmacro{\rot}{\dy > 0 ? 90 : -90}
  \else
    \pgfmathsetmacro{\rot}{\dx > 0 ? atan(\dy / \dx) : 180 + atan(\dy / \dx)}
  \fi

  
   
  \shadedraw[transform canvas = { rotate around = {\rot:(\sx,\sy)}}, line width = \linewidth, rounded corners = \radius mm, top color = \ofill, bottom color = \ofill, middle color = \ifill, draw = \stroke] ($ (start)+(-0.5*\height, 0.5*\height) $) -- ++(\height cm +\length pt, 0 ) -- ++(0, -\height) -- ++ (-\height cm -\length pt, 0) -- cycle;


  \shadedraw[ball color = \ofill!50!\ifill, draw = \stroke] (start) circle (\height/8);
  \shadedraw[ball color = \ofill!50!\ifill, draw = \stroke] (end) circle (\height/8);
  %  \pgfresetboundingbox

  
  


}


\newcommand{\PC}[6][0]{%
  \edef\lrotate{#1}%
  \edef\lpin{#2}%
  \edef\lfill{#3}%
  \edef\ldraw{#4}%
  \edef\lscale{#5}%
  \edef\lwidth{#6}%
  \edef\h{1}%
  \edef\r{0.3}%
  \begin{scope}[scale=\lscale, rotate=\lrotate]
	\filldraw[draw=\ldraw, fill=\lfill, line width=\lwidth mm] ($ (\lpin) + (0.201*\h+1.0353*\r ,-0.75*\h) $) -- ++(105: 0.77646*\h+0.26795*\r) arc (15:165:\r) -- ++(-105:0.77646*\h+0.26795*\r) -- cycle;

	\shadedraw[ball color=\lfill, draw=\ldraw, line width = \lwidth mm] (\lpin) circle (1.5mm);

	\filldraw[rounded corners=\lscale pt, draw=\ldraw, fill=\lfill, line width=\lwidth mm] ($ (\lpin) - (1,1) $) rectangle +(2,0.25);
  \end{scope}%
}




\usepackage{mathpazo}
\usepackage[letterpaper]{geometry}
\geometry{verbose,tmargin=0.25in,bmargin=0.5in,lmargin=1in,rmargin=1.15in}
\hfuzz 50pt
\setlength{\parindent}{0pt}
\Large

\begin{document}

%%%%%%%%%%%%%%%%%%%%%%%%%%%%%%%%%%%%%%%%%%%%%%%%%%%%%%%%%%%%%%%%%%%%%%%%%%%%%%%%%%%%%%%%%%%%%%%%%%%%
% page 1
%%%%%%%%%%%%%%%%%%%%%%%%%%%%%%%%%%%%%%%%%%%%%%%%%%%%%%%%%%%%%%%%%%%%%%%%%%%%%%%%%%%%%%%%%%%%%%%%%%%%



% \begin{textblock*}{8in}(0in, 0.25in)
	
	% \cbox{
		
		% \mini[0.5]{
			% \cbox{
			\centering
			\tikz{%color
			\coordinate (O) at (0,0);
			
			\filldraw[fill=Gray0!30, draw=black, thick] (O) -- +(5,0) arc (0:180:5)-- cycle;
			\filldraw[DarkOliveGreen2, draw=Black, thick] ($ (O)+(30:5) $) arc (30:60:5) -- ($ (O)+(120:5) $) arc (120:150:5) -- cycle;
			\filldraw[Gold2, draw=Black, thick] ($ (O)+(60:5) $) arc (60:120:5) -- cycle;
			
			\fill (O) circle (1mm) node[below left] {$O$};
			\draw ($ (O)+(0,-.5) $) -- +(0,6);
			\draw (O) -- +(30:5);
			\draw (O) -- +(60:5);
			\draw ($ (O)+(14:4.8) $) -- ($ (O)+(14:5.2) $) ;
			\draw ($ (O)+(16:4.8) $) -- ($ (O)+(16:5.2) $) ;
			\draw ($ (O)+(44:4.8) $) -- ($ (O)+(44:5.2) $) ;
			\draw ($ (O)+(46:4.8) $) -- ($ (O)+(46:5.2) $) ;
			\draw ($ (O)+(74:4.8) $) -- ($ (O)+(74:5.2) $) ;
			\draw ($ (O)+(76:4.8) $) -- ($ (O)+(76:5.2) $) ;

			\node at ($ (O)+(14:1.25) $) {$30\deg$};
			\node at ($ (O)+(42:1.25) $) {$30\deg$};
			\node at ($ (O)+(72:1.25) $) {$30\deg$};

			\node at ($ (O)+(150:5.325) $) {$A$};
			\node at ($ (O)+(30:5.325) $) {$A'$};
			\node at ($ (O)+(120:5.325) $) {$B$};
			\node at ($ (O)+(60:5.325) $) {$B'$};
		}
		
	%  }}
	%  \hfill
	 \cmini[0.6]{
		% \cbox{
		The 'gold' area is given by 
		$$ A = \frac{(\theta-\sin\theta)D^2}{8} $$
		where $\theta$ is the angle subtended at $O$ by $BB'$, in radians ($\pi/3$) and $D$ is the diameter ($ 12  $). 
		\begin{align*}
			A_{Gold} &= \frac{\left(\frac{\pi}{3}-\sin 60\deg\right)(24)^2}{8} \\
				&= 13.044
		\end{align*}
		(Yes, I know I'm mixing units here but $\sin 60\deg=\sin \frac{\pi}{3}$ radians and I like to keep my calculator in degree mode -- because I have a habit of forgetting to return to degree mode...)
		\parm
		Similarly, the 'gold and green' areas are given by:
		\begin{align*}
			A_{Green+Gold} &= \frac{\left(\frac{2\pi}{3}-\sin 120\deg\right)(24)^2}{8} \\
				&= 88.443
		\end{align*}
		So, the 'green' area ($AA'B'B$) is simply the difference between the two areas:
		$$ A_{Green} = 88.443-13.044 = 75.399 $$
		But the original question only wants half of this so the answer is 37.7 units.
	
		% }
	 }
		\newpage
	 \tikz[scale=1]{
			\coordinate (O) at (0,0);
			
			\filldraw[fill=Gray0!30, draw=black, thick] (O) -- node[midway, below]{12} +(5,0) arc (0:90:5)-- cycle;
			\filldraw[DarkOliveGreen2, draw=Black, thick] ($ (O)+(30:5) $) arc (30:60:5) -- (0, 4.331) --(0,2.5) -- cycle;
			
			
			\fill (O) circle (1mm) node[below left] {$O$};
			\draw ($ (O)+(0,-.5) $) -- +(0,6);
			\draw (O) -- +(30:5);
			\draw (O) -- +(60:5);
			\draw ($ (O)+(14:4.8) $) -- ($ (O)+(14:5.2) $) ;
			\draw ($ (O)+(16:4.8) $) -- ($ (O)+(16:5.2) $) ;
			\draw ($ (O)+(44:4.8) $) -- ($ (O)+(44:5.2) $) ;
			\draw ($ (O)+(46:4.8) $) -- ($ (O)+(46:5.2) $) ;
			\draw ($ (O)+(74:4.8) $) -- ($ (O)+(74:5.2) $) ;
			\draw ($ (O)+(76:4.8) $) -- ($ (O)+(76:5.2) $) ;

			\node at ($ (O)+(15:1.125) $) {$\frac{\pi}{6}$};
			\node at ($ (O)+(45:1.125) $) {$\frac{\pi}{6}$};
			\node at ($ (O)+(75:1.125) $) {$\frac{\pi}{6}$};
			\node at ($ (O)+(35:3.8) $) {$\frac{\pi}{6}$};
			\node at ($ (O)+(53:2.675) $) {$\frac{2\pi}{3}$};

			\draw[thick] ($ (0.945,1.75)+(150:0.1) $) -- ($ (0.945,1.75)+(150:-0.2) $);
			\draw[thick] ($ (47:3.4)+(0,0.15) $) -- ($ (47:3.4)+(0,-0.15) $);

			\node[below] at ($ (O)+(5,0) $) {$A$};
			\node at ($ (O)+(30:5.325) $) {$B$};
			\node at ($ (O)+(60:5.325) $) {$C$};
			\node[left] at ($ (O)+(90:4.331) $) {$D$};
			\node[left] at ($ (O)+(90:2.5) $) {$E$};
			\node[below] at ($ (O)+(66:3.25) $) {$F$};
			
			\node[below] at ($ (O)+(50:4.5) $) {$\bm A_1$};
			\node[below] at ($ (O)+(75:4) $) {$\bm A_2$};
		}
		\cmini[0.6]{
			% Or, more geometrically:
			\begin{align*} 
				\bm{A_1} &= \text{Area }OBC - \text{Area }OBF \\
				\text{Area }OBC &= \frac{\pi/6}{2\pi}\cdot \pi(12)^2 = 12\pi \\
				\intertext{For $\triangle OBF$:}			
				\frac{BF}{\sin \pi/6} &= \frac{12}{\sin 2\pi/3} \\
				\Rightarrow \frac{BF}{1/2} &= \frac{12}{\sqrt{3}/2} \\
				\Rightarrow BF &= \frac{12}{\sqrt{3}} \\
				OE &= 12\cos \pi/3 = 6 \\
				\Rightarrow \text{Area }\triangle OBF &= \frac 12\cd\left(\frac{12}{\sqrt{3}}\right)\cd 6= \frac{36}{\sqrt{3}}
				\intertext{Thus:}
				\bm{A_1} &= 12\pi-\frac{36}{\sqrt{3}}
				\intertext{Now, for $\bm{A_2}$}			
				OD &= 12\cos \pi/6 = 12\cd \frac{\sqrt{3}}{2}=6\sqrt{3} \\
				\Rightarrow DE &= OD-OE = 6\sqrt{3}-6 \\
				CD &= 12\sin \pi/6 = 6 \\[0.125cm]
				BE &= 12\cos \pi/6 = 6\sqrt{3} \\
				\Rightarrow \text{Area }\triangle OCD &= \frac 12\cd CD \cd OD = \frac 12 \cd 6\cd 6\sqrt{3}=18\sqrt{3} \\
				\text{ and area }\triangle OEF &= \frac 12\cd EF \cd OE = \frac 12 \cd \left(6\sqrt{3}-\frac{12}{\sqrt{3}}\right)\cd 6 \\
				&= 18\sqrt{3}- \frac{36}{\sqrt{3}} \\
				\Rightarrow \bm{A_2} &= 18\sqrt{3}-18\sqrt{3}+\frac{36}{\sqrt{3}}= \frac{36}{\sqrt{3}}\\
				\text{Finally, } \bm{A_1+A_2} &= 12\pi - \frac{36}{\sqrt{3}}+\frac{36}{\sqrt{3}}=\bm{12\pi}
			\end{align*}



		}

\end{document}